
    




    
\documentclass[11pt]{article}

    
    \usepackage[breakable]{tcolorbox}
    \tcbset{nobeforeafter} % prevents tcolorboxes being placing in paragraphs
    \usepackage{float}
    \floatplacement{figure}{H} % forces figures to be placed at the correct location
    
    \usepackage[T1]{fontenc}
    % Nicer default font (+ math font) than Computer Modern for most use cases
    \usepackage{mathpazo}

    % Basic figure setup, for now with no caption control since it's done
    % automatically by Pandoc (which extracts ![](path) syntax from Markdown).
    \usepackage{graphicx}
    % We will generate all images so they have a width \maxwidth. This means
    % that they will get their normal width if they fit onto the page, but
    % are scaled down if they would overflow the margins.
    \makeatletter
    \def\maxwidth{\ifdim\Gin@nat@width>\linewidth\linewidth
    \else\Gin@nat@width\fi}
    \makeatother
    \let\Oldincludegraphics\includegraphics
    % Set max figure width to be 80% of text width, for now hardcoded.
    \renewcommand{\includegraphics}[1]{\Oldincludegraphics[width=.8\maxwidth]{#1}}
    % Ensure that by default, figures have no caption (until we provide a
    % proper Figure object with a Caption API and a way to capture that
    % in the conversion process - todo).
    \usepackage{caption}
    \DeclareCaptionLabelFormat{nolabel}{}
    \captionsetup{labelformat=nolabel}

    \usepackage{adjustbox} % Used to constrain images to a maximum size 
    \usepackage{xcolor} % Allow colors to be defined
    \usepackage{enumerate} % Needed for markdown enumerations to work
    \usepackage{geometry} % Used to adjust the document margins
    \usepackage{amsmath} % Equations
    \usepackage{amssymb} % Equations
    \usepackage{textcomp} % defines textquotesingle
    % Hack from http://tex.stackexchange.com/a/47451/13684:
    \AtBeginDocument{%
        \def\PYZsq{\textquotesingle}% Upright quotes in Pygmentized code
    }
    \usepackage{upquote} % Upright quotes for verbatim code
    \usepackage{eurosym} % defines \euro
    \usepackage[mathletters]{ucs} % Extended unicode (utf-8) support
    \usepackage[utf8x]{inputenc} % Allow utf-8 characters in the tex document
    \usepackage{fancyvrb} % verbatim replacement that allows latex
    \usepackage{grffile} % extends the file name processing of package graphics 
                         % to support a larger range 
    % The hyperref package gives us a pdf with properly built
    % internal navigation ('pdf bookmarks' for the table of contents,
    % internal cross-reference links, web links for URLs, etc.)
    \usepackage{hyperref}
    \usepackage{longtable} % longtable support required by pandoc >1.10
    \usepackage{booktabs}  % table support for pandoc > 1.12.2
    \usepackage[inline]{enumitem} % IRkernel/repr support (it uses the enumerate* environment)
    \usepackage[normalem]{ulem} % ulem is needed to support strikethroughs (\sout)
                                % normalem makes italics be italics, not underlines
    \usepackage{mathrsfs}
    

    
    % Colors for the hyperref package
    \definecolor{urlcolor}{rgb}{0,.145,.698}
    \definecolor{linkcolor}{rgb}{.71,0.21,0.01}
    \definecolor{citecolor}{rgb}{.12,.54,.11}

    % ANSI colors
    \definecolor{ansi-black}{HTML}{3E424D}
    \definecolor{ansi-black-intense}{HTML}{282C36}
    \definecolor{ansi-red}{HTML}{E75C58}
    \definecolor{ansi-red-intense}{HTML}{B22B31}
    \definecolor{ansi-green}{HTML}{00A250}
    \definecolor{ansi-green-intense}{HTML}{007427}
    \definecolor{ansi-yellow}{HTML}{DDB62B}
    \definecolor{ansi-yellow-intense}{HTML}{B27D12}
    \definecolor{ansi-blue}{HTML}{208FFB}
    \definecolor{ansi-blue-intense}{HTML}{0065CA}
    \definecolor{ansi-magenta}{HTML}{D160C4}
    \definecolor{ansi-magenta-intense}{HTML}{A03196}
    \definecolor{ansi-cyan}{HTML}{60C6C8}
    \definecolor{ansi-cyan-intense}{HTML}{258F8F}
    \definecolor{ansi-white}{HTML}{C5C1B4}
    \definecolor{ansi-white-intense}{HTML}{A1A6B2}
    \definecolor{ansi-default-inverse-fg}{HTML}{FFFFFF}
    \definecolor{ansi-default-inverse-bg}{HTML}{000000}

    % commands and environments needed by pandoc snippets
    % extracted from the output of `pandoc -s`
    \providecommand{\tightlist}{%
      \setlength{\itemsep}{0pt}\setlength{\parskip}{0pt}}
    \DefineVerbatimEnvironment{Highlighting}{Verbatim}{commandchars=\\\{\}}
    % Add ',fontsize=\small' for more characters per line
    \newenvironment{Shaded}{}{}
    \newcommand{\KeywordTok}[1]{\textcolor[rgb]{0.00,0.44,0.13}{\textbf{{#1}}}}
    \newcommand{\DataTypeTok}[1]{\textcolor[rgb]{0.56,0.13,0.00}{{#1}}}
    \newcommand{\DecValTok}[1]{\textcolor[rgb]{0.25,0.63,0.44}{{#1}}}
    \newcommand{\BaseNTok}[1]{\textcolor[rgb]{0.25,0.63,0.44}{{#1}}}
    \newcommand{\FloatTok}[1]{\textcolor[rgb]{0.25,0.63,0.44}{{#1}}}
    \newcommand{\CharTok}[1]{\textcolor[rgb]{0.25,0.44,0.63}{{#1}}}
    \newcommand{\StringTok}[1]{\textcolor[rgb]{0.25,0.44,0.63}{{#1}}}
    \newcommand{\CommentTok}[1]{\textcolor[rgb]{0.38,0.63,0.69}{\textit{{#1}}}}
    \newcommand{\OtherTok}[1]{\textcolor[rgb]{0.00,0.44,0.13}{{#1}}}
    \newcommand{\AlertTok}[1]{\textcolor[rgb]{1.00,0.00,0.00}{\textbf{{#1}}}}
    \newcommand{\FunctionTok}[1]{\textcolor[rgb]{0.02,0.16,0.49}{{#1}}}
    \newcommand{\RegionMarkerTok}[1]{{#1}}
    \newcommand{\ErrorTok}[1]{\textcolor[rgb]{1.00,0.00,0.00}{\textbf{{#1}}}}
    \newcommand{\NormalTok}[1]{{#1}}
    
    % Additional commands for more recent versions of Pandoc
    \newcommand{\ConstantTok}[1]{\textcolor[rgb]{0.53,0.00,0.00}{{#1}}}
    \newcommand{\SpecialCharTok}[1]{\textcolor[rgb]{0.25,0.44,0.63}{{#1}}}
    \newcommand{\VerbatimStringTok}[1]{\textcolor[rgb]{0.25,0.44,0.63}{{#1}}}
    \newcommand{\SpecialStringTok}[1]{\textcolor[rgb]{0.73,0.40,0.53}{{#1}}}
    \newcommand{\ImportTok}[1]{{#1}}
    \newcommand{\DocumentationTok}[1]{\textcolor[rgb]{0.73,0.13,0.13}{\textit{{#1}}}}
    \newcommand{\AnnotationTok}[1]{\textcolor[rgb]{0.38,0.63,0.69}{\textbf{\textit{{#1}}}}}
    \newcommand{\CommentVarTok}[1]{\textcolor[rgb]{0.38,0.63,0.69}{\textbf{\textit{{#1}}}}}
    \newcommand{\VariableTok}[1]{\textcolor[rgb]{0.10,0.09,0.49}{{#1}}}
    \newcommand{\ControlFlowTok}[1]{\textcolor[rgb]{0.00,0.44,0.13}{\textbf{{#1}}}}
    \newcommand{\OperatorTok}[1]{\textcolor[rgb]{0.40,0.40,0.40}{{#1}}}
    \newcommand{\BuiltInTok}[1]{{#1}}
    \newcommand{\ExtensionTok}[1]{{#1}}
    \newcommand{\PreprocessorTok}[1]{\textcolor[rgb]{0.74,0.48,0.00}{{#1}}}
    \newcommand{\AttributeTok}[1]{\textcolor[rgb]{0.49,0.56,0.16}{{#1}}}
    \newcommand{\InformationTok}[1]{\textcolor[rgb]{0.38,0.63,0.69}{\textbf{\textit{{#1}}}}}
    \newcommand{\WarningTok}[1]{\textcolor[rgb]{0.38,0.63,0.69}{\textbf{\textit{{#1}}}}}
    
    
    % Define a nice break command that doesn't care if a line doesn't already
    % exist.
    \def\br{\hspace*{\fill} \\* }
    % Math Jax compatibility definitions
    \def\gt{>}
    \def\lt{<}
    \let\Oldtex\TeX
    \let\Oldlatex\LaTeX
    \renewcommand{\TeX}{\textrm{\Oldtex}}
    \renewcommand{\LaTeX}{\textrm{\Oldlatex}}
    % Document parameters
    % Document title
    \title{probttr}
    
    
    
    
    
% Pygments definitions
\makeatletter
\def\PY@reset{\let\PY@it=\relax \let\PY@bf=\relax%
    \let\PY@ul=\relax \let\PY@tc=\relax%
    \let\PY@bc=\relax \let\PY@ff=\relax}
\def\PY@tok#1{\csname PY@tok@#1\endcsname}
\def\PY@toks#1+{\ifx\relax#1\empty\else%
    \PY@tok{#1}\expandafter\PY@toks\fi}
\def\PY@do#1{\PY@bc{\PY@tc{\PY@ul{%
    \PY@it{\PY@bf{\PY@ff{#1}}}}}}}
\def\PY#1#2{\PY@reset\PY@toks#1+\relax+\PY@do{#2}}

\expandafter\def\csname PY@tok@w\endcsname{\def\PY@tc##1{\textcolor[rgb]{0.73,0.73,0.73}{##1}}}
\expandafter\def\csname PY@tok@c\endcsname{\let\PY@it=\textit\def\PY@tc##1{\textcolor[rgb]{0.25,0.50,0.50}{##1}}}
\expandafter\def\csname PY@tok@cp\endcsname{\def\PY@tc##1{\textcolor[rgb]{0.74,0.48,0.00}{##1}}}
\expandafter\def\csname PY@tok@k\endcsname{\let\PY@bf=\textbf\def\PY@tc##1{\textcolor[rgb]{0.00,0.50,0.00}{##1}}}
\expandafter\def\csname PY@tok@kp\endcsname{\def\PY@tc##1{\textcolor[rgb]{0.00,0.50,0.00}{##1}}}
\expandafter\def\csname PY@tok@kt\endcsname{\def\PY@tc##1{\textcolor[rgb]{0.69,0.00,0.25}{##1}}}
\expandafter\def\csname PY@tok@o\endcsname{\def\PY@tc##1{\textcolor[rgb]{0.40,0.40,0.40}{##1}}}
\expandafter\def\csname PY@tok@ow\endcsname{\let\PY@bf=\textbf\def\PY@tc##1{\textcolor[rgb]{0.67,0.13,1.00}{##1}}}
\expandafter\def\csname PY@tok@nb\endcsname{\def\PY@tc##1{\textcolor[rgb]{0.00,0.50,0.00}{##1}}}
\expandafter\def\csname PY@tok@nf\endcsname{\def\PY@tc##1{\textcolor[rgb]{0.00,0.00,1.00}{##1}}}
\expandafter\def\csname PY@tok@nc\endcsname{\let\PY@bf=\textbf\def\PY@tc##1{\textcolor[rgb]{0.00,0.00,1.00}{##1}}}
\expandafter\def\csname PY@tok@nn\endcsname{\let\PY@bf=\textbf\def\PY@tc##1{\textcolor[rgb]{0.00,0.00,1.00}{##1}}}
\expandafter\def\csname PY@tok@ne\endcsname{\let\PY@bf=\textbf\def\PY@tc##1{\textcolor[rgb]{0.82,0.25,0.23}{##1}}}
\expandafter\def\csname PY@tok@nv\endcsname{\def\PY@tc##1{\textcolor[rgb]{0.10,0.09,0.49}{##1}}}
\expandafter\def\csname PY@tok@no\endcsname{\def\PY@tc##1{\textcolor[rgb]{0.53,0.00,0.00}{##1}}}
\expandafter\def\csname PY@tok@nl\endcsname{\def\PY@tc##1{\textcolor[rgb]{0.63,0.63,0.00}{##1}}}
\expandafter\def\csname PY@tok@ni\endcsname{\let\PY@bf=\textbf\def\PY@tc##1{\textcolor[rgb]{0.60,0.60,0.60}{##1}}}
\expandafter\def\csname PY@tok@na\endcsname{\def\PY@tc##1{\textcolor[rgb]{0.49,0.56,0.16}{##1}}}
\expandafter\def\csname PY@tok@nt\endcsname{\let\PY@bf=\textbf\def\PY@tc##1{\textcolor[rgb]{0.00,0.50,0.00}{##1}}}
\expandafter\def\csname PY@tok@nd\endcsname{\def\PY@tc##1{\textcolor[rgb]{0.67,0.13,1.00}{##1}}}
\expandafter\def\csname PY@tok@s\endcsname{\def\PY@tc##1{\textcolor[rgb]{0.73,0.13,0.13}{##1}}}
\expandafter\def\csname PY@tok@sd\endcsname{\let\PY@it=\textit\def\PY@tc##1{\textcolor[rgb]{0.73,0.13,0.13}{##1}}}
\expandafter\def\csname PY@tok@si\endcsname{\let\PY@bf=\textbf\def\PY@tc##1{\textcolor[rgb]{0.73,0.40,0.53}{##1}}}
\expandafter\def\csname PY@tok@se\endcsname{\let\PY@bf=\textbf\def\PY@tc##1{\textcolor[rgb]{0.73,0.40,0.13}{##1}}}
\expandafter\def\csname PY@tok@sr\endcsname{\def\PY@tc##1{\textcolor[rgb]{0.73,0.40,0.53}{##1}}}
\expandafter\def\csname PY@tok@ss\endcsname{\def\PY@tc##1{\textcolor[rgb]{0.10,0.09,0.49}{##1}}}
\expandafter\def\csname PY@tok@sx\endcsname{\def\PY@tc##1{\textcolor[rgb]{0.00,0.50,0.00}{##1}}}
\expandafter\def\csname PY@tok@m\endcsname{\def\PY@tc##1{\textcolor[rgb]{0.40,0.40,0.40}{##1}}}
\expandafter\def\csname PY@tok@gh\endcsname{\let\PY@bf=\textbf\def\PY@tc##1{\textcolor[rgb]{0.00,0.00,0.50}{##1}}}
\expandafter\def\csname PY@tok@gu\endcsname{\let\PY@bf=\textbf\def\PY@tc##1{\textcolor[rgb]{0.50,0.00,0.50}{##1}}}
\expandafter\def\csname PY@tok@gd\endcsname{\def\PY@tc##1{\textcolor[rgb]{0.63,0.00,0.00}{##1}}}
\expandafter\def\csname PY@tok@gi\endcsname{\def\PY@tc##1{\textcolor[rgb]{0.00,0.63,0.00}{##1}}}
\expandafter\def\csname PY@tok@gr\endcsname{\def\PY@tc##1{\textcolor[rgb]{1.00,0.00,0.00}{##1}}}
\expandafter\def\csname PY@tok@ge\endcsname{\let\PY@it=\textit}
\expandafter\def\csname PY@tok@gs\endcsname{\let\PY@bf=\textbf}
\expandafter\def\csname PY@tok@gp\endcsname{\let\PY@bf=\textbf\def\PY@tc##1{\textcolor[rgb]{0.00,0.00,0.50}{##1}}}
\expandafter\def\csname PY@tok@go\endcsname{\def\PY@tc##1{\textcolor[rgb]{0.53,0.53,0.53}{##1}}}
\expandafter\def\csname PY@tok@gt\endcsname{\def\PY@tc##1{\textcolor[rgb]{0.00,0.27,0.87}{##1}}}
\expandafter\def\csname PY@tok@err\endcsname{\def\PY@bc##1{\setlength{\fboxsep}{0pt}\fcolorbox[rgb]{1.00,0.00,0.00}{1,1,1}{\strut ##1}}}
\expandafter\def\csname PY@tok@kc\endcsname{\let\PY@bf=\textbf\def\PY@tc##1{\textcolor[rgb]{0.00,0.50,0.00}{##1}}}
\expandafter\def\csname PY@tok@kd\endcsname{\let\PY@bf=\textbf\def\PY@tc##1{\textcolor[rgb]{0.00,0.50,0.00}{##1}}}
\expandafter\def\csname PY@tok@kn\endcsname{\let\PY@bf=\textbf\def\PY@tc##1{\textcolor[rgb]{0.00,0.50,0.00}{##1}}}
\expandafter\def\csname PY@tok@kr\endcsname{\let\PY@bf=\textbf\def\PY@tc##1{\textcolor[rgb]{0.00,0.50,0.00}{##1}}}
\expandafter\def\csname PY@tok@bp\endcsname{\def\PY@tc##1{\textcolor[rgb]{0.00,0.50,0.00}{##1}}}
\expandafter\def\csname PY@tok@fm\endcsname{\def\PY@tc##1{\textcolor[rgb]{0.00,0.00,1.00}{##1}}}
\expandafter\def\csname PY@tok@vc\endcsname{\def\PY@tc##1{\textcolor[rgb]{0.10,0.09,0.49}{##1}}}
\expandafter\def\csname PY@tok@vg\endcsname{\def\PY@tc##1{\textcolor[rgb]{0.10,0.09,0.49}{##1}}}
\expandafter\def\csname PY@tok@vi\endcsname{\def\PY@tc##1{\textcolor[rgb]{0.10,0.09,0.49}{##1}}}
\expandafter\def\csname PY@tok@vm\endcsname{\def\PY@tc##1{\textcolor[rgb]{0.10,0.09,0.49}{##1}}}
\expandafter\def\csname PY@tok@sa\endcsname{\def\PY@tc##1{\textcolor[rgb]{0.73,0.13,0.13}{##1}}}
\expandafter\def\csname PY@tok@sb\endcsname{\def\PY@tc##1{\textcolor[rgb]{0.73,0.13,0.13}{##1}}}
\expandafter\def\csname PY@tok@sc\endcsname{\def\PY@tc##1{\textcolor[rgb]{0.73,0.13,0.13}{##1}}}
\expandafter\def\csname PY@tok@dl\endcsname{\def\PY@tc##1{\textcolor[rgb]{0.73,0.13,0.13}{##1}}}
\expandafter\def\csname PY@tok@s2\endcsname{\def\PY@tc##1{\textcolor[rgb]{0.73,0.13,0.13}{##1}}}
\expandafter\def\csname PY@tok@sh\endcsname{\def\PY@tc##1{\textcolor[rgb]{0.73,0.13,0.13}{##1}}}
\expandafter\def\csname PY@tok@s1\endcsname{\def\PY@tc##1{\textcolor[rgb]{0.73,0.13,0.13}{##1}}}
\expandafter\def\csname PY@tok@mb\endcsname{\def\PY@tc##1{\textcolor[rgb]{0.40,0.40,0.40}{##1}}}
\expandafter\def\csname PY@tok@mf\endcsname{\def\PY@tc##1{\textcolor[rgb]{0.40,0.40,0.40}{##1}}}
\expandafter\def\csname PY@tok@mh\endcsname{\def\PY@tc##1{\textcolor[rgb]{0.40,0.40,0.40}{##1}}}
\expandafter\def\csname PY@tok@mi\endcsname{\def\PY@tc##1{\textcolor[rgb]{0.40,0.40,0.40}{##1}}}
\expandafter\def\csname PY@tok@il\endcsname{\def\PY@tc##1{\textcolor[rgb]{0.40,0.40,0.40}{##1}}}
\expandafter\def\csname PY@tok@mo\endcsname{\def\PY@tc##1{\textcolor[rgb]{0.40,0.40,0.40}{##1}}}
\expandafter\def\csname PY@tok@ch\endcsname{\let\PY@it=\textit\def\PY@tc##1{\textcolor[rgb]{0.25,0.50,0.50}{##1}}}
\expandafter\def\csname PY@tok@cm\endcsname{\let\PY@it=\textit\def\PY@tc##1{\textcolor[rgb]{0.25,0.50,0.50}{##1}}}
\expandafter\def\csname PY@tok@cpf\endcsname{\let\PY@it=\textit\def\PY@tc##1{\textcolor[rgb]{0.25,0.50,0.50}{##1}}}
\expandafter\def\csname PY@tok@c1\endcsname{\let\PY@it=\textit\def\PY@tc##1{\textcolor[rgb]{0.25,0.50,0.50}{##1}}}
\expandafter\def\csname PY@tok@cs\endcsname{\let\PY@it=\textit\def\PY@tc##1{\textcolor[rgb]{0.25,0.50,0.50}{##1}}}

\def\PYZbs{\char`\\}
\def\PYZus{\char`\_}
\def\PYZob{\char`\{}
\def\PYZcb{\char`\}}
\def\PYZca{\char`\^}
\def\PYZam{\char`\&}
\def\PYZlt{\char`\<}
\def\PYZgt{\char`\>}
\def\PYZsh{\char`\#}
\def\PYZpc{\char`\%}
\def\PYZdl{\char`\$}
\def\PYZhy{\char`\-}
\def\PYZsq{\char`\'}
\def\PYZdq{\char`\"}
\def\PYZti{\char`\~}
% for compatibility with earlier versions
\def\PYZat{@}
\def\PYZlb{[}
\def\PYZrb{]}
\makeatother


    % For linebreaks inside Verbatim environment from package fancyvrb. 
    \makeatletter
        \newbox\Wrappedcontinuationbox 
        \newbox\Wrappedvisiblespacebox 
        \newcommand*\Wrappedvisiblespace {\textcolor{red}{\textvisiblespace}} 
        \newcommand*\Wrappedcontinuationsymbol {\textcolor{red}{\llap{\tiny$\m@th\hookrightarrow$}}} 
        \newcommand*\Wrappedcontinuationindent {3ex } 
        \newcommand*\Wrappedafterbreak {\kern\Wrappedcontinuationindent\copy\Wrappedcontinuationbox} 
        % Take advantage of the already applied Pygments mark-up to insert 
        % potential linebreaks for TeX processing. 
        %        {, <, #, %, $, ' and ": go to next line. 
        %        _, }, ^, &, >, - and ~: stay at end of broken line. 
        % Use of \textquotesingle for straight quote. 
        \newcommand*\Wrappedbreaksatspecials {% 
            \def\PYGZus{\discretionary{\char`\_}{\Wrappedafterbreak}{\char`\_}}% 
            \def\PYGZob{\discretionary{}{\Wrappedafterbreak\char`\{}{\char`\{}}% 
            \def\PYGZcb{\discretionary{\char`\}}{\Wrappedafterbreak}{\char`\}}}% 
            \def\PYGZca{\discretionary{\char`\^}{\Wrappedafterbreak}{\char`\^}}% 
            \def\PYGZam{\discretionary{\char`\&}{\Wrappedafterbreak}{\char`\&}}% 
            \def\PYGZlt{\discretionary{}{\Wrappedafterbreak\char`\<}{\char`\<}}% 
            \def\PYGZgt{\discretionary{\char`\>}{\Wrappedafterbreak}{\char`\>}}% 
            \def\PYGZsh{\discretionary{}{\Wrappedafterbreak\char`\#}{\char`\#}}% 
            \def\PYGZpc{\discretionary{}{\Wrappedafterbreak\char`\%}{\char`\%}}% 
            \def\PYGZdl{\discretionary{}{\Wrappedafterbreak\char`\$}{\char`\$}}% 
            \def\PYGZhy{\discretionary{\char`\-}{\Wrappedafterbreak}{\char`\-}}% 
            \def\PYGZsq{\discretionary{}{\Wrappedafterbreak\textquotesingle}{\textquotesingle}}% 
            \def\PYGZdq{\discretionary{}{\Wrappedafterbreak\char`\"}{\char`\"}}% 
            \def\PYGZti{\discretionary{\char`\~}{\Wrappedafterbreak}{\char`\~}}% 
        } 
        % Some characters . , ; ? ! / are not pygmentized. 
        % This macro makes them "active" and they will insert potential linebreaks 
        \newcommand*\Wrappedbreaksatpunct {% 
            \lccode`\~`\.\lowercase{\def~}{\discretionary{\hbox{\char`\.}}{\Wrappedafterbreak}{\hbox{\char`\.}}}% 
            \lccode`\~`\,\lowercase{\def~}{\discretionary{\hbox{\char`\,}}{\Wrappedafterbreak}{\hbox{\char`\,}}}% 
            \lccode`\~`\;\lowercase{\def~}{\discretionary{\hbox{\char`\;}}{\Wrappedafterbreak}{\hbox{\char`\;}}}% 
            \lccode`\~`\:\lowercase{\def~}{\discretionary{\hbox{\char`\:}}{\Wrappedafterbreak}{\hbox{\char`\:}}}% 
            \lccode`\~`\?\lowercase{\def~}{\discretionary{\hbox{\char`\?}}{\Wrappedafterbreak}{\hbox{\char`\?}}}% 
            \lccode`\~`\!\lowercase{\def~}{\discretionary{\hbox{\char`\!}}{\Wrappedafterbreak}{\hbox{\char`\!}}}% 
            \lccode`\~`\/\lowercase{\def~}{\discretionary{\hbox{\char`\/}}{\Wrappedafterbreak}{\hbox{\char`\/}}}% 
            \catcode`\.\active
            \catcode`\,\active 
            \catcode`\;\active
            \catcode`\:\active
            \catcode`\?\active
            \catcode`\!\active
            \catcode`\/\active 
            \lccode`\~`\~ 	
        }
    \makeatother

    \let\OriginalVerbatim=\Verbatim
    \makeatletter
    \renewcommand{\Verbatim}[1][1]{%
        %\parskip\z@skip
        \sbox\Wrappedcontinuationbox {\Wrappedcontinuationsymbol}%
        \sbox\Wrappedvisiblespacebox {\FV@SetupFont\Wrappedvisiblespace}%
        \def\FancyVerbFormatLine ##1{\hsize\linewidth
            \vtop{\raggedright\hyphenpenalty\z@\exhyphenpenalty\z@
                \doublehyphendemerits\z@\finalhyphendemerits\z@
                \strut ##1\strut}%
        }%
        % If the linebreak is at a space, the latter will be displayed as visible
        % space at end of first line, and a continuation symbol starts next line.
        % Stretch/shrink are however usually zero for typewriter font.
        \def\FV@Space {%
            \nobreak\hskip\z@ plus\fontdimen3\font minus\fontdimen4\font
            \discretionary{\copy\Wrappedvisiblespacebox}{\Wrappedafterbreak}
            {\kern\fontdimen2\font}%
        }%
        
        % Allow breaks at special characters using \PYG... macros.
        \Wrappedbreaksatspecials
        % Breaks at punctuation characters . , ; ? ! and / need catcode=\active 	
        \OriginalVerbatim[#1,codes*=\Wrappedbreaksatpunct]%
    }
    \makeatother

    % Exact colors from NB
    \definecolor{incolor}{HTML}{303F9F}
    \definecolor{outcolor}{HTML}{D84315}
    \definecolor{cellborder}{HTML}{CFCFCF}
    \definecolor{cellbackground}{HTML}{F7F7F7}
    
    % prompt
    \newcommand{\prompt}[4]{
        \llap{{\color{#2}[#3]: #4}}\vspace{-1.25em}
    }
    

    
    % Prevent overflowing lines due to hard-to-break entities
    \sloppy 
    % Setup hyperref package
    \hypersetup{
      breaklinks=true,  % so long urls are correctly broken across lines
      colorlinks=true,
      urlcolor=urlcolor,
      linkcolor=linkcolor,
      citecolor=citecolor,
      }
    % Slightly bigger margins than the latex defaults
    
    \geometry{verbose,tmargin=1in,bmargin=1in,lmargin=1in,rmargin=1in}
    
    

    \begin{document}
    
    
    \maketitle
    
    

    
    \begin{tcolorbox}[breakable, size=fbox, boxrule=1pt, pad at break*=1mm,colback=cellbackground, colframe=cellborder]
\prompt{In}{incolor}{1}{\hspace{4pt}}
\begin{Verbatim}[commandchars=\\\{\}]
\PY{k+kn}{from} \PY{n+nn}{probttrtypes} \PY{k}{import} \PY{n}{Type}\PY{p}{,} \PY{n}{PConstraint}\PY{p}{,} \PY{n}{BType}\PY{p}{,} \PY{n}{PType}\PY{p}{,} \PY{n}{Pred}\PY{p}{,}\PYZbs{}
\PY{n}{Possibility}\PY{p}{,} \PY{n}{MeetType}\PY{p}{,} \PY{n}{JoinType}\PY{p}{,} \PY{n}{RecType}\PY{p}{,} \PY{n}{Fun}
\PY{k+kn}{from} \PY{n+nn}{utils} \PY{k}{import} \PY{n}{show}\PY{p}{,} \PY{n}{show\PYZus{}latex}\PY{p}{,} \PY{n}{ttrace}\PY{p}{,} \PY{n}{nottrace}
\PY{k+kn}{from} \PY{n+nn}{records} \PY{k}{import} \PY{n}{Rec}
\end{Verbatim}
\end{tcolorbox}

    \hypertarget{judging-probabilities}{%
\section{Judging probabilities}\label{judging-probabilities}}

    The witness cache in \texttt{probttr} is a pair whose first member is a
list of objects and whose second member is a list of probabilities
(actually probability constraints)

    \begin{tcolorbox}[breakable, size=fbox, boxrule=1pt, pad at break*=1mm,colback=cellbackground, colframe=cellborder]
\prompt{In}{incolor}{2}{\hspace{4pt}}
\begin{Verbatim}[commandchars=\\\{\}]
\PY{n}{T} \PY{o}{=} \PY{n}{Type}\PY{p}{(}\PY{p}{)}
\PY{n+nb}{print}\PY{p}{(}\PY{n}{T}\PY{o}{.}\PY{n}{witness\PYZus{}cache}\PY{p}{)}
\end{Verbatim}
\end{tcolorbox}

    \begin{Verbatim}[commandchars=\\\{\}]
([], [])
\end{Verbatim}

    \begin{tcolorbox}[breakable, size=fbox, boxrule=1pt, pad at break*=1mm,colback=cellbackground, colframe=cellborder]
\prompt{In}{incolor}{3}{\hspace{4pt}}
\begin{Verbatim}[commandchars=\\\{\}]
\PY{n}{show}\PY{p}{(}\PY{n}{T}\PY{o}{.}\PY{n}{judge}\PY{p}{(}\PY{l+s+s1}{\PYZsq{}}\PY{l+s+s1}{a}\PY{l+s+s1}{\PYZsq{}}\PY{p}{,}\PY{o}{.}\PY{l+m+mi}{5}\PY{p}{)}\PY{p}{)}
\end{Verbatim}
\end{tcolorbox}

            \begin{tcolorbox}[breakable, boxrule=.5pt, size=fbox, pad at break*=1mm, opacityfill=0]
\prompt{Out}{outcolor}{3}{\hspace{3.5pt}}
\begin{Verbatim}[commandchars=\\\{\}]
'0.5'
\end{Verbatim}
\end{tcolorbox}
        
    \begin{tcolorbox}[breakable, size=fbox, boxrule=1pt, pad at break*=1mm,colback=cellbackground, colframe=cellborder]
\prompt{In}{incolor}{4}{\hspace{4pt}}
\begin{Verbatim}[commandchars=\\\{\}]
\PY{n}{show}\PY{p}{(}\PY{n}{T}\PY{o}{.}\PY{n}{witness\PYZus{}cache}\PY{p}{)}
\end{Verbatim}
\end{tcolorbox}

            \begin{tcolorbox}[breakable, boxrule=.5pt, size=fbox, pad at break*=1mm, opacityfill=0]
\prompt{Out}{outcolor}{4}{\hspace{3.5pt}}
\begin{Verbatim}[commandchars=\\\{\}]
'([a], [0.5])'
\end{Verbatim}
\end{tcolorbox}
        
    \begin{tcolorbox}[breakable, size=fbox, boxrule=1pt, pad at break*=1mm,colback=cellbackground, colframe=cellborder]
\prompt{In}{incolor}{5}{\hspace{4pt}}
\begin{Verbatim}[commandchars=\\\{\}]
\PY{n}{show}\PY{p}{(}\PY{n}{T}\PY{o}{.}\PY{n}{judge}\PY{p}{(}\PY{l+s+s1}{\PYZsq{}}\PY{l+s+s1}{a}\PY{l+s+s1}{\PYZsq{}}\PY{p}{,}\PY{o}{.}\PY{l+m+mi}{6}\PY{p}{)}\PY{p}{)}
\end{Verbatim}
\end{tcolorbox}

            \begin{tcolorbox}[breakable, boxrule=.5pt, size=fbox, pad at break*=1mm, opacityfill=0]
\prompt{Out}{outcolor}{5}{\hspace{3.5pt}}
\begin{Verbatim}[commandchars=\\\{\}]
'0.6'
\end{Verbatim}
\end{tcolorbox}
        
    \begin{tcolorbox}[breakable, size=fbox, boxrule=1pt, pad at break*=1mm,colback=cellbackground, colframe=cellborder]
\prompt{In}{incolor}{6}{\hspace{4pt}}
\begin{Verbatim}[commandchars=\\\{\}]
\PY{n}{show}\PY{p}{(}\PY{n}{T}\PY{o}{.}\PY{n}{witness\PYZus{}cache}\PY{p}{)}
\end{Verbatim}
\end{tcolorbox}

            \begin{tcolorbox}[breakable, boxrule=.5pt, size=fbox, pad at break*=1mm, opacityfill=0]
\prompt{Out}{outcolor}{6}{\hspace{3.5pt}}
\begin{Verbatim}[commandchars=\\\{\}]
'([a], [0.6])'
\end{Verbatim}
\end{tcolorbox}
        
    Adding an additional probability argument to \texttt{judge()} gives you
a minimum and maximum probability constraint.

    \begin{tcolorbox}[breakable, size=fbox, boxrule=1pt, pad at break*=1mm,colback=cellbackground, colframe=cellborder]
\prompt{In}{incolor}{7}{\hspace{4pt}}
\begin{Verbatim}[commandchars=\\\{\}]
\PY{n}{show}\PY{p}{(}\PY{n}{T}\PY{o}{.}\PY{n}{judge}\PY{p}{(}\PY{l+s+s1}{\PYZsq{}}\PY{l+s+s1}{a}\PY{l+s+s1}{\PYZsq{}}\PY{p}{,}\PY{o}{.}\PY{l+m+mi}{6}\PY{p}{,}\PY{l+m+mi}{1}\PY{p}{)}\PY{p}{)}
\end{Verbatim}
\end{tcolorbox}

            \begin{tcolorbox}[breakable, boxrule=.5pt, size=fbox, pad at break*=1mm, opacityfill=0]
\prompt{Out}{outcolor}{7}{\hspace{3.5pt}}
\begin{Verbatim}[commandchars=\\\{\}]
'>=0.6'
\end{Verbatim}
\end{tcolorbox}
        
    \begin{tcolorbox}[breakable, size=fbox, boxrule=1pt, pad at break*=1mm,colback=cellbackground, colframe=cellborder]
\prompt{In}{incolor}{8}{\hspace{4pt}}
\begin{Verbatim}[commandchars=\\\{\}]
\PY{n}{show\PYZus{}latex}\PY{p}{(}\PY{n}{T}\PY{o}{.}\PY{n}{judge}\PY{p}{(}\PY{l+s+s1}{\PYZsq{}}\PY{l+s+s1}{a}\PY{l+s+s1}{\PYZsq{}}\PY{p}{,}\PY{o}{.}\PY{l+m+mi}{6}\PY{p}{,}\PY{o}{.}\PY{l+m+mi}{8}\PY{p}{)}\PY{p}{)}
\end{Verbatim}
\end{tcolorbox}
 
            
\prompt{Out}{outcolor}{8}{}
    
    \begin{equation}\geq0.6\&\leq0.8\end{equation}

    

    \begin{tcolorbox}[breakable, size=fbox, boxrule=1pt, pad at break*=1mm,colback=cellbackground, colframe=cellborder]
\prompt{In}{incolor}{9}{\hspace{4pt}}
\begin{Verbatim}[commandchars=\\\{\}]
\PY{n}{show\PYZus{}latex}\PY{p}{(}\PY{n}{T}\PY{o}{.}\PY{n}{judge}\PY{p}{(}\PY{l+s+s1}{\PYZsq{}}\PY{l+s+s1}{a}\PY{l+s+s1}{\PYZsq{}}\PY{p}{,}\PY{l+m+mi}{0}\PY{p}{,}\PY{o}{.}\PY{l+m+mi}{6}\PY{p}{)}\PY{p}{)}
\end{Verbatim}
\end{tcolorbox}
 
            
\prompt{Out}{outcolor}{9}{}
    
    \begin{equation}\leq0.6\end{equation}

    

    \texttt{judge(a,n,n)} is the same as \texttt{judge(a,n)}

    \begin{tcolorbox}[breakable, size=fbox, boxrule=1pt, pad at break*=1mm,colback=cellbackground, colframe=cellborder]
\prompt{In}{incolor}{10}{\hspace{4pt}}
\begin{Verbatim}[commandchars=\\\{\}]
\PY{n}{show\PYZus{}latex}\PY{p}{(}\PY{n}{T}\PY{o}{.}\PY{n}{judge}\PY{p}{(}\PY{l+s+s1}{\PYZsq{}}\PY{l+s+s1}{a}\PY{l+s+s1}{\PYZsq{}}\PY{p}{,}\PY{o}{.}\PY{l+m+mi}{6}\PY{p}{,}\PY{o}{.}\PY{l+m+mi}{6}\PY{p}{)}\PY{p}{)}
\end{Verbatim}
\end{tcolorbox}
 
            
\prompt{Out}{outcolor}{10}{}
    
    \begin{equation}0.6\end{equation}

    

    Probabilities must be between 0 and 1. This can be checked by using the
method \texttt{validate()} on a probability constraint.

    \begin{tcolorbox}[breakable, size=fbox, boxrule=1pt, pad at break*=1mm,colback=cellbackground, colframe=cellborder]
\prompt{In}{incolor}{11}{\hspace{4pt}}
\begin{Verbatim}[commandchars=\\\{\}]
\PY{n+nb}{print}\PY{p}{(}\PY{n}{T}\PY{o}{.}\PY{n}{judge}\PY{p}{(}\PY{l+s+s1}{\PYZsq{}}\PY{l+s+s1}{a}\PY{l+s+s1}{\PYZsq{}}\PY{p}{,}\PY{o}{\PYZhy{}}\PY{l+m+mi}{1}\PY{p}{)}\PY{o}{.}\PY{n}{validate}\PY{p}{(}\PY{p}{)}\PY{p}{)}
\PY{n+nb}{print}\PY{p}{(}\PY{n}{T}\PY{o}{.}\PY{n}{judge}\PY{p}{(}\PY{l+s+s1}{\PYZsq{}}\PY{l+s+s1}{a}\PY{l+s+s1}{\PYZsq{}}\PY{p}{,}\PY{o}{.}\PY{l+m+mi}{6}\PY{p}{,}\PY{o}{.}\PY{l+m+mi}{1}\PY{p}{)}\PY{o}{.}\PY{n}{validate}\PY{p}{(}\PY{p}{)}\PY{p}{)}
\end{Verbatim}
\end{tcolorbox}

    \begin{Verbatim}[commandchars=\\\{\}]
-1.0 is less than 0.
False
0.6 is greater than 0.1
False
\end{Verbatim}

    In contrast to non-probabilistic \texttt{pyttr} we can store a negative
judgement in witness cache.

    \begin{tcolorbox}[breakable, size=fbox, boxrule=1pt, pad at break*=1mm,colback=cellbackground, colframe=cellborder]
\prompt{In}{incolor}{12}{\hspace{4pt}}
\begin{Verbatim}[commandchars=\\\{\}]
\PY{n}{T}\PY{o}{.}\PY{n}{judge}\PY{p}{(}\PY{l+s+s1}{\PYZsq{}}\PY{l+s+s1}{a}\PY{l+s+s1}{\PYZsq{}}\PY{p}{,}\PY{l+m+mi}{0}\PY{p}{)}
\PY{n}{show\PYZus{}latex}\PY{p}{(}\PY{n}{T}\PY{o}{.}\PY{n}{witness\PYZus{}cache}\PY{p}{)}
\end{Verbatim}
\end{tcolorbox}
 
            
\prompt{Out}{outcolor}{12}{}
    
    \begin{equation}\langle [ \text{a}], [ 0.0]\rangle\end{equation}

    

    For compatibility with non-probabilistic TTR: \texttt{judge(a)} is the
same as \texttt{judge(a,1)} (which is the same as
\texttt{judge(a,1,1)}).

    \begin{tcolorbox}[breakable, size=fbox, boxrule=1pt, pad at break*=1mm,colback=cellbackground, colframe=cellborder]
\prompt{In}{incolor}{13}{\hspace{4pt}}
\begin{Verbatim}[commandchars=\\\{\}]
\PY{n}{show}\PY{p}{(}\PY{n}{T}\PY{o}{.}\PY{n}{judge}\PY{p}{(}\PY{l+s+s1}{\PYZsq{}}\PY{l+s+s1}{a}\PY{l+s+s1}{\PYZsq{}}\PY{p}{)}\PY{p}{)}
\end{Verbatim}
\end{tcolorbox}

            \begin{tcolorbox}[breakable, boxrule=.5pt, size=fbox, pad at break*=1mm, opacityfill=0]
\prompt{Out}{outcolor}{13}{\hspace{3.5pt}}
\begin{Verbatim}[commandchars=\\\{\}]
'1.0'
\end{Verbatim}
\end{tcolorbox}
        
    \hypertarget{non-specific-judging}{%
\subsection{Non-specific judging}\label{non-specific-judging}}

    We can also make judgements about the probability that something belongs
to a type using the method \texttt{judge\_nonspec}.

    \begin{tcolorbox}[breakable, size=fbox, boxrule=1pt, pad at break*=1mm,colback=cellbackground, colframe=cellborder]
\prompt{In}{incolor}{14}{\hspace{4pt}}
\begin{Verbatim}[commandchars=\\\{\}]
\PY{n}{T\PYZus{}new} \PY{o}{=} \PY{n}{Type}\PY{p}{(}\PY{p}{)}
\PY{n}{T\PYZus{}new}\PY{o}{.}\PY{n}{judge\PYZus{}nonspec}\PY{p}{(}\PY{o}{.}\PY{l+m+mi}{3}\PY{p}{,}\PY{o}{.}\PY{l+m+mi}{4}\PY{p}{)}
\PY{n}{show}\PY{p}{(}\PY{n}{T\PYZus{}new}\PY{o}{.}\PY{n}{prob\PYZus{}nonspec}\PY{p}{)}
\end{Verbatim}
\end{tcolorbox}

            \begin{tcolorbox}[breakable, boxrule=.5pt, size=fbox, pad at break*=1mm, opacityfill=0]
\prompt{Out}{outcolor}{14}{\hspace{3.5pt}}
\begin{Verbatim}[commandchars=\\\{\}]
'>=0.3\&<=0.4'
\end{Verbatim}
\end{tcolorbox}
        
    \hypertarget{querying-probabilities}{%
\section{Querying probabilities}\label{querying-probabilities}}

    \hypertarget{querying-unconditional-probabilities}{%
\subsection{Querying unconditional
probabilities}\label{querying-unconditional-probabilities}}

    If \texttt{a} is in the witness cache, \texttt{query(a)} returns the
probability stored in the witness cache for \texttt{a}.

    \begin{tcolorbox}[breakable, size=fbox, boxrule=1pt, pad at break*=1mm,colback=cellbackground, colframe=cellborder]
\prompt{In}{incolor}{15}{\hspace{4pt}}
\begin{Verbatim}[commandchars=\\\{\}]
\PY{n}{show}\PY{p}{(}\PY{n}{T}\PY{o}{.}\PY{n}{query}\PY{p}{(}\PY{l+s+s1}{\PYZsq{}}\PY{l+s+s1}{a}\PY{l+s+s1}{\PYZsq{}}\PY{p}{)}\PY{p}{)}
\end{Verbatim}
\end{tcolorbox}

            \begin{tcolorbox}[breakable, boxrule=.5pt, size=fbox, pad at break*=1mm, opacityfill=0]
\prompt{Out}{outcolor}{15}{\hspace{3.5pt}}
\begin{Verbatim}[commandchars=\\\{\}]
'1.0'
\end{Verbatim}
\end{tcolorbox}
        
    If \texttt{a} is not in the witness cache and we have no other way of
computing a probability that \texttt{a} is a witness for \texttt{T},
then the probability range is \texttt{{[}0,1{]}}, i.e.~\(\leq 1\). This
corresponds to returning an answer \texttt{Don\textquotesingle{}t\ know}
or \texttt{Undecided}.

    \begin{tcolorbox}[breakable, size=fbox, boxrule=1pt, pad at break*=1mm,colback=cellbackground, colframe=cellborder]
\prompt{In}{incolor}{16}{\hspace{4pt}}
\begin{Verbatim}[commandchars=\\\{\}]
\PY{n}{show}\PY{p}{(}\PY{n}{T}\PY{o}{.}\PY{n}{query}\PY{p}{(}\PY{l+s+s1}{\PYZsq{}}\PY{l+s+s1}{b}\PY{l+s+s1}{\PYZsq{}}\PY{p}{)}\PY{p}{)}
\end{Verbatim}
\end{tcolorbox}

            \begin{tcolorbox}[breakable, boxrule=.5pt, size=fbox, pad at break*=1mm, opacityfill=0]
\prompt{Out}{outcolor}{16}{\hspace{3.5pt}}
\begin{Verbatim}[commandchars=\\\{\}]
'<=1.0'
\end{Verbatim}
\end{tcolorbox}
        
    \texttt{Don\textquotesingle{}t\ know} results are not added to the
witness cache. (This may or may not be a good idea.)

    \begin{tcolorbox}[breakable, size=fbox, boxrule=1pt, pad at break*=1mm,colback=cellbackground, colframe=cellborder]
\prompt{In}{incolor}{17}{\hspace{4pt}}
\begin{Verbatim}[commandchars=\\\{\}]
\PY{n}{show}\PY{p}{(}\PY{n}{T}\PY{o}{.}\PY{n}{witness\PYZus{}cache}\PY{p}{)}
\end{Verbatim}
\end{tcolorbox}

            \begin{tcolorbox}[breakable, boxrule=.5pt, size=fbox, pad at break*=1mm, opacityfill=0]
\prompt{Out}{outcolor}{17}{\hspace{3.5pt}}
\begin{Verbatim}[commandchars=\\\{\}]
'([a], [1.0])'
\end{Verbatim}
\end{tcolorbox}
        
    Witness conditions are functions which return probability constraints.
Here is type \texttt{Real} for real numbers, implemented as floating
point decimals where the witness condition gives a categorical judgement
for any object, that is it returns probability 1 or 0.

    \begin{tcolorbox}[breakable, size=fbox, boxrule=1pt, pad at break*=1mm,colback=cellbackground, colframe=cellborder]
\prompt{In}{incolor}{18}{\hspace{4pt}}
\begin{Verbatim}[commandchars=\\\{\}]
\PY{k}{def} \PY{n+nf}{RealClassifier}\PY{p}{(}\PY{n}{n}\PY{p}{)}\PY{p}{:}
    \PY{k}{if} \PY{n+nb}{isinstance}\PY{p}{(}\PY{n}{n}\PY{p}{,}\PY{n+nb}{float}\PY{p}{)}\PY{p}{:}
        \PY{k}{return} \PY{n}{PConstraint}\PY{p}{(}\PY{l+m+mi}{1}\PY{p}{)}
    \PY{k}{else}\PY{p}{:}
        \PY{k}{return} \PY{n}{PConstraint}\PY{p}{(}\PY{l+m+mi}{0}\PY{p}{)}
\PY{n}{Real} \PY{o}{=} \PY{n}{Type}\PY{p}{(}\PY{l+s+s1}{\PYZsq{}}\PY{l+s+s1}{Real}\PY{l+s+s1}{\PYZsq{}}\PY{p}{)}
\PY{n}{Real}\PY{o}{.}\PY{n}{learn\PYZus{}witness\PYZus{}condition}\PY{p}{(}\PY{n}{RealClassifier}\PY{p}{)}
\PY{n}{show}\PY{p}{(}\PY{n}{Real}\PY{o}{.}\PY{n}{query}\PY{p}{(}\PY{o}{.}\PY{l+m+mi}{6}\PY{p}{)}\PY{p}{)}
\end{Verbatim}
\end{tcolorbox}

            \begin{tcolorbox}[breakable, boxrule=.5pt, size=fbox, pad at break*=1mm, opacityfill=0]
\prompt{Out}{outcolor}{18}{\hspace{3.5pt}}
\begin{Verbatim}[commandchars=\\\{\}]
'1.0'
\end{Verbatim}
\end{tcolorbox}
        
    \begin{tcolorbox}[breakable, size=fbox, boxrule=1pt, pad at break*=1mm,colback=cellbackground, colframe=cellborder]
\prompt{In}{incolor}{19}{\hspace{4pt}}
\begin{Verbatim}[commandchars=\\\{\}]
\PY{n}{show}\PY{p}{(}\PY{n}{Real}\PY{o}{.}\PY{n}{query}\PY{p}{(}\PY{l+s+s1}{\PYZsq{}}\PY{l+s+s1}{a}\PY{l+s+s1}{\PYZsq{}}\PY{p}{)}\PY{p}{)}
\end{Verbatim}
\end{tcolorbox}

            \begin{tcolorbox}[breakable, boxrule=.5pt, size=fbox, pad at break*=1mm, opacityfill=0]
\prompt{Out}{outcolor}{19}{\hspace{3.5pt}}
\begin{Verbatim}[commandchars=\\\{\}]
'0.0'
\end{Verbatim}
\end{tcolorbox}
        
    \begin{tcolorbox}[breakable, size=fbox, boxrule=1pt, pad at break*=1mm,colback=cellbackground, colframe=cellborder]
\prompt{In}{incolor}{20}{\hspace{4pt}}
\begin{Verbatim}[commandchars=\\\{\}]
\PY{n}{show}\PY{p}{(}\PY{n}{Real}\PY{o}{.}\PY{n}{witness\PYZus{}cache}\PY{p}{)}
\end{Verbatim}
\end{tcolorbox}

            \begin{tcolorbox}[breakable, boxrule=.5pt, size=fbox, pad at break*=1mm, opacityfill=0]
\prompt{Out}{outcolor}{20}{\hspace{3.5pt}}
\begin{Verbatim}[commandchars=\\\{\}]
'([0.6, a], [1.0, 0.0])'
\end{Verbatim}
\end{tcolorbox}
        
    Here's an example with a couple of witness conditions which return
probability constraints. \texttt{query(a)} returns the maximum obtained
by the witness conditions for \texttt{a}. Note that this need not be
identical with either of the constraints returned by the individual
witness conditions since the maximum of a collection of probability
constraints is defined has as minimum the maximum of all the minima and
as maximum the maximum of all the maxima. (It could, of course, be done
differently\ldots)

    \begin{tcolorbox}[breakable, size=fbox, boxrule=1pt, pad at break*=1mm,colback=cellbackground, colframe=cellborder]
\prompt{In}{incolor}{21}{\hspace{4pt}}
\begin{Verbatim}[commandchars=\\\{\}]
\PY{n}{T\PYZus{}at} \PY{o}{=} \PY{n}{Type}\PY{p}{(}\PY{l+s+s1}{\PYZsq{}}\PY{l+s+s1}{T\PYZus{}at}\PY{l+s+s1}{\PYZsq{}}\PY{p}{)}
\PY{k}{def} \PY{n+nf}{Classifier\PYZus{}a}\PY{p}{(}\PY{n}{s}\PY{p}{)}\PY{p}{:}
    \PY{k}{if} \PY{l+s+s1}{\PYZsq{}}\PY{l+s+s1}{a}\PY{l+s+s1}{\PYZsq{}} \PY{o+ow}{in} \PY{n}{s}\PY{p}{:}
        \PY{k}{return} \PY{n}{PConstraint}\PY{p}{(}\PY{o}{.}\PY{l+m+mi}{8}\PY{p}{,}\PY{o}{.}\PY{l+m+mi}{9}\PY{p}{)}
    \PY{k}{else}\PY{p}{:}
        \PY{k}{return} \PY{n}{PConstraint}\PY{p}{(}\PY{o}{.}\PY{l+m+mi}{1}\PY{p}{,}\PY{o}{.}\PY{l+m+mi}{3}\PY{p}{)}
\PY{k}{def} \PY{n+nf}{Classifier\PYZus{}t}\PY{p}{(}\PY{n}{s}\PY{p}{)}\PY{p}{:}
    \PY{k}{if} \PY{l+s+s1}{\PYZsq{}}\PY{l+s+s1}{t}\PY{l+s+s1}{\PYZsq{}} \PY{o+ow}{in} \PY{n}{s}\PY{p}{:}
        \PY{k}{return} \PY{n}{PConstraint}\PY{p}{(}\PY{o}{.}\PY{l+m+mi}{2}\PY{p}{,}\PY{o}{.}\PY{l+m+mi}{95}\PY{p}{)}
    \PY{k}{else}\PY{p}{:}
        \PY{k}{return} \PY{n}{PConstraint}\PY{p}{(}\PY{o}{.}\PY{l+m+mi}{15}\PY{p}{,}\PY{o}{.}\PY{l+m+mi}{7}\PY{p}{)}
\PY{n}{T\PYZus{}at}\PY{o}{.}\PY{n}{learn\PYZus{}witness\PYZus{}condition}\PY{p}{(}\PY{n}{Classifier\PYZus{}a}\PY{p}{)}
\PY{n}{T\PYZus{}at}\PY{o}{.}\PY{n}{learn\PYZus{}witness\PYZus{}condition}\PY{p}{(}\PY{n}{Classifier\PYZus{}t}\PY{p}{)}
\PY{n}{show}\PY{p}{(}\PY{n}{T\PYZus{}at}\PY{o}{.}\PY{n}{query}\PY{p}{(}\PY{l+s+s1}{\PYZsq{}}\PY{l+s+s1}{a}\PY{l+s+s1}{\PYZsq{}}\PY{p}{)}\PY{p}{)}
\end{Verbatim}
\end{tcolorbox}

            \begin{tcolorbox}[breakable, boxrule=.5pt, size=fbox, pad at break*=1mm, opacityfill=0]
\prompt{Out}{outcolor}{21}{\hspace{3.5pt}}
\begin{Verbatim}[commandchars=\\\{\}]
'>=0.8\&<=0.9'
\end{Verbatim}
\end{tcolorbox}
        
    \begin{tcolorbox}[breakable, size=fbox, boxrule=1pt, pad at break*=1mm,colback=cellbackground, colframe=cellborder]
\prompt{In}{incolor}{22}{\hspace{4pt}}
\begin{Verbatim}[commandchars=\\\{\}]
\PY{n}{show}\PY{p}{(}\PY{n}{T\PYZus{}at}\PY{o}{.}\PY{n}{query}\PY{p}{(}\PY{l+s+s1}{\PYZsq{}}\PY{l+s+s1}{t}\PY{l+s+s1}{\PYZsq{}}\PY{p}{)}\PY{p}{)}
\end{Verbatim}
\end{tcolorbox}

            \begin{tcolorbox}[breakable, boxrule=.5pt, size=fbox, pad at break*=1mm, opacityfill=0]
\prompt{Out}{outcolor}{22}{\hspace{3.5pt}}
\begin{Verbatim}[commandchars=\\\{\}]
'>=0.2\&<=0.95'
\end{Verbatim}
\end{tcolorbox}
        
    \begin{tcolorbox}[breakable, size=fbox, boxrule=1pt, pad at break*=1mm,colback=cellbackground, colframe=cellborder]
\prompt{In}{incolor}{23}{\hspace{4pt}}
\begin{Verbatim}[commandchars=\\\{\}]
\PY{n}{show}\PY{p}{(}\PY{n}{T\PYZus{}at}\PY{o}{.}\PY{n}{query}\PY{p}{(}\PY{l+s+s1}{\PYZsq{}}\PY{l+s+s1}{at}\PY{l+s+s1}{\PYZsq{}}\PY{p}{)}\PY{p}{)}
\end{Verbatim}
\end{tcolorbox}

            \begin{tcolorbox}[breakable, boxrule=.5pt, size=fbox, pad at break*=1mm, opacityfill=0]
\prompt{Out}{outcolor}{23}{\hspace{3.5pt}}
\begin{Verbatim}[commandchars=\\\{\}]
'>=0.8\&<=0.95'
\end{Verbatim}
\end{tcolorbox}
        
    \begin{tcolorbox}[breakable, size=fbox, boxrule=1pt, pad at break*=1mm,colback=cellbackground, colframe=cellborder]
\prompt{In}{incolor}{24}{\hspace{4pt}}
\begin{Verbatim}[commandchars=\\\{\}]
\PY{n}{show}\PY{p}{(}\PY{n}{T\PYZus{}at}\PY{o}{.}\PY{n}{query}\PY{p}{(}\PY{l+s+s1}{\PYZsq{}}\PY{l+s+s1}{b}\PY{l+s+s1}{\PYZsq{}}\PY{p}{)}\PY{p}{)}
\end{Verbatim}
\end{tcolorbox}

            \begin{tcolorbox}[breakable, boxrule=.5pt, size=fbox, pad at break*=1mm, opacityfill=0]
\prompt{Out}{outcolor}{24}{\hspace{3.5pt}}
\begin{Verbatim}[commandchars=\\\{\}]
'>=0.15\&<=0.7'
\end{Verbatim}
\end{tcolorbox}
        
    \hypertarget{querying-conditional-probabilities}{%
\subsection{Querying conditional
probabilities}\label{querying-conditional-probabilities}}

    Conditions are provided as a second argument to the \texttt{query()}
method as a list each of whose members is \emph{either} a tuple,
\texttt{(a,T)}, where \texttt{a} is an object and \texttt{T} is a type,
\emph{or} a type. The idea is that, for example,
\texttt{T.query(a,{[}(b,T1),T2{]})} queries the probability that
\texttt{a} is of type \texttt{T} given that \texttt{b} is of type
\texttt{T1} and there is some witness for \texttt{T2}.

    Consider a query \texttt{T.query(a,Conds)}. If the probability that
\texttt{a} is of type \texttt{T} does not depend on any of the
conditions in \texttt{Conds} then what is returned is the same as
\texttt{T.query(a)}, that is the unconditional probability. The default
assumption is that probabilities are independent.

    \begin{tcolorbox}[breakable, size=fbox, boxrule=1pt, pad at break*=1mm,colback=cellbackground, colframe=cellborder]
\prompt{In}{incolor}{25}{\hspace{4pt}}
\begin{Verbatim}[commandchars=\\\{\}]
\PY{n}{T1} \PY{o}{=} \PY{n}{Type}\PY{p}{(}\PY{p}{)}
\PY{n}{T2} \PY{o}{=} \PY{n}{Type}\PY{p}{(}\PY{p}{)}
\PY{n}{T1}\PY{o}{.}\PY{n}{judge}\PY{p}{(}\PY{l+s+s1}{\PYZsq{}}\PY{l+s+s1}{a}\PY{l+s+s1}{\PYZsq{}}\PY{p}{,}\PY{o}{.}\PY{l+m+mi}{6}\PY{p}{)}
\PY{n}{show}\PY{p}{(}\PY{n}{T1}\PY{o}{.}\PY{n}{query}\PY{p}{(}\PY{l+s+s1}{\PYZsq{}}\PY{l+s+s1}{a}\PY{l+s+s1}{\PYZsq{}}\PY{p}{,}\PY{p}{[}\PY{p}{(}\PY{l+s+s1}{\PYZsq{}}\PY{l+s+s1}{b}\PY{l+s+s1}{\PYZsq{}}\PY{p}{,}\PY{n}{T2}\PY{p}{)}\PY{p}{]}\PY{p}{)}\PY{p}{)}
\end{Verbatim}
\end{tcolorbox}

            \begin{tcolorbox}[breakable, boxrule=.5pt, size=fbox, pad at break*=1mm, opacityfill=0]
\prompt{Out}{outcolor}{25}{\hspace{3.5pt}}
\begin{Verbatim}[commandchars=\\\{\}]
'0.6'
\end{Verbatim}
\end{tcolorbox}
        
    One kind of dependency between probabilities relates to subtyping in the
type theory. Suppose that \(T_2\sqsubseteq T_1\). Then
\(p(a:T_1|a:T_2)=1\). A limit case of this is where we have the same
type judgement in the conditions as the one we are querying:
\(p(a:T|a:T)\), that is, the probability that \(a\) is of type \(T\)
given that \(a\) is of type \(T\) has to be \(1\), no matter what the
unconditional probability is that \(a\) is of type \(T\).

    \begin{tcolorbox}[breakable, size=fbox, boxrule=1pt, pad at break*=1mm,colback=cellbackground, colframe=cellborder]
\prompt{In}{incolor}{26}{\hspace{4pt}}
\begin{Verbatim}[commandchars=\\\{\}]
\PY{n}{show}\PY{p}{(}\PY{n}{T1}\PY{o}{.}\PY{n}{query}\PY{p}{(}\PY{l+s+s1}{\PYZsq{}}\PY{l+s+s1}{a}\PY{l+s+s1}{\PYZsq{}}\PY{p}{,}\PY{p}{[}\PY{p}{(}\PY{l+s+s1}{\PYZsq{}}\PY{l+s+s1}{a}\PY{l+s+s1}{\PYZsq{}}\PY{p}{,}\PY{n}{T1}\PY{p}{)}\PY{p}{]}\PY{p}{)}\PY{p}{)}
\end{Verbatim}
\end{tcolorbox}

            \begin{tcolorbox}[breakable, boxrule=.5pt, size=fbox, pad at break*=1mm, opacityfill=0]
\prompt{Out}{outcolor}{26}{\hspace{3.5pt}}
\begin{Verbatim}[commandchars=\\\{\}]
'1.0'
\end{Verbatim}
\end{tcolorbox}
        
    Let us now create a type \texttt{T3} which is a subtype of \texttt{T1}

    \begin{tcolorbox}[breakable, size=fbox, boxrule=1pt, pad at break*=1mm,colback=cellbackground, colframe=cellborder]
\prompt{In}{incolor}{27}{\hspace{4pt}}
\begin{Verbatim}[commandchars=\\\{\}]
\PY{n}{T3} \PY{o}{=} \PY{n}{Type}\PY{p}{(}\PY{p}{)}
\PY{n}{T1}\PY{o}{.}\PY{n}{learn\PYZus{}witness\PYZus{}condition}\PY{p}{(}\PY{k}{lambda} \PY{n}{x}\PY{p}{:} \PY{n}{T3}\PY{o}{.}\PY{n}{query}\PY{p}{(}\PY{n}{x}\PY{p}{)}\PY{p}{)}
\PY{n}{T3}\PY{o}{.}\PY{n}{subtype\PYZus{}of}\PY{p}{(}\PY{n}{T1}\PY{p}{)}
\end{Verbatim}
\end{tcolorbox}

            \begin{tcolorbox}[breakable, boxrule=.5pt, size=fbox, pad at break*=1mm, opacityfill=0]
\prompt{Out}{outcolor}{27}{\hspace{3.5pt}}
\begin{Verbatim}[commandchars=\\\{\}]
True
\end{Verbatim}
\end{tcolorbox}
        
    What is the probability that something is of type \texttt{T1} given that
it is of type \texttt{T3}?

    \begin{tcolorbox}[breakable, size=fbox, boxrule=1pt, pad at break*=1mm,colback=cellbackground, colframe=cellborder]
\prompt{In}{incolor}{28}{\hspace{4pt}}
\begin{Verbatim}[commandchars=\\\{\}]
\PY{n}{show}\PY{p}{(}\PY{n}{T1}\PY{o}{.}\PY{n}{query}\PY{p}{(}\PY{l+s+s1}{\PYZsq{}}\PY{l+s+s1}{b}\PY{l+s+s1}{\PYZsq{}}\PY{p}{,}\PY{p}{[}\PY{p}{(}\PY{l+s+s1}{\PYZsq{}}\PY{l+s+s1}{b}\PY{l+s+s1}{\PYZsq{}}\PY{p}{,}\PY{n}{T3}\PY{p}{)}\PY{p}{]}\PY{p}{)}\PY{p}{)}
\end{Verbatim}
\end{tcolorbox}

            \begin{tcolorbox}[breakable, boxrule=.5pt, size=fbox, pad at break*=1mm, opacityfill=0]
\prompt{Out}{outcolor}{28}{\hspace{3.5pt}}
\begin{Verbatim}[commandchars=\\\{\}]
'1.0'
\end{Verbatim}
\end{tcolorbox}
        
    Dependent probabilities that are not related to subtyping have to be
provided by an oracle which is given as a third argument to
\texttt{query()}. An oracle is a python function which takes three
arguments: an object, a type and a list of conditions of the kind which
is used as an argument to \texttt{query()}. For any such argument it
returns either a probability constraint (e.g.~\texttt{PConstraint(.6)})
or \texttt{None}. As a python function the oracle may call on resources
external to \texttt{pyttr}, for example, conditional probability tables
or Bayesian networks. A call \texttt{T.query(a,c,o)} where \texttt{c}
does not contain \texttt{(a,TT)} where \texttt{TT} is a subtype of
\texttt{T} will return \texttt{o(a,T,c)} if this is not \texttt{None}.
Otherwise \texttt{T.query(a,c,o)} will return \texttt{T.query(a)} (the
unconditional probability). If \texttt{c} \emph{does} contain
\texttt{(a,TT)} where \texttt{TT} is a subtype of \texttt{T}, then the
oracle will be ignored and \texttt{T.query(a,c,o)} will return
\texttt{PConstraint(1)}.

    As an example we define a trivial oracle.

    \begin{tcolorbox}[breakable, size=fbox, boxrule=1pt, pad at break*=1mm,colback=cellbackground, colframe=cellborder]
\prompt{In}{incolor}{29}{\hspace{4pt}}
\begin{Verbatim}[commandchars=\\\{\}]
\PY{k}{def} \PY{n+nf}{SillyOracle}\PY{p}{(}\PY{n}{a}\PY{p}{,}\PY{n}{T}\PY{p}{,}\PY{n}{c}\PY{p}{)}\PY{p}{:}
    \PY{k}{if} \PY{n}{a} \PY{o+ow}{is} \PY{l+s+s1}{\PYZsq{}}\PY{l+s+s1}{a}\PY{l+s+s1}{\PYZsq{}}\PY{o+ow}{and} \PY{n}{T} \PY{o+ow}{is} \PY{n}{T1} \PY{o+ow}{and} \PY{p}{(}\PY{l+s+s1}{\PYZsq{}}\PY{l+s+s1}{b}\PY{l+s+s1}{\PYZsq{}}\PY{p}{,}\PY{n}{T2}\PY{p}{)} \PY{o+ow}{in} \PY{n}{c}\PY{p}{:}
        \PY{k}{return} \PY{n}{PConstraint}\PY{p}{(}\PY{o}{.}\PY{l+m+mi}{7}\PY{p}{,}\PY{o}{.}\PY{l+m+mi}{8}\PY{p}{)}
    \PY{k}{else}\PY{p}{:}
        \PY{k}{return}
\end{Verbatim}
\end{tcolorbox}

    Using the oracle.

    \begin{tcolorbox}[breakable, size=fbox, boxrule=1pt, pad at break*=1mm,colback=cellbackground, colframe=cellborder]
\prompt{In}{incolor}{30}{\hspace{4pt}}
\begin{Verbatim}[commandchars=\\\{\}]
\PY{n}{show}\PY{p}{(}\PY{n}{T1}\PY{o}{.}\PY{n}{query}\PY{p}{(}\PY{l+s+s1}{\PYZsq{}}\PY{l+s+s1}{a}\PY{l+s+s1}{\PYZsq{}}\PY{p}{,}\PY{p}{[}\PY{p}{(}\PY{l+s+s1}{\PYZsq{}}\PY{l+s+s1}{b}\PY{l+s+s1}{\PYZsq{}}\PY{p}{,}\PY{n}{T2}\PY{p}{)}\PY{p}{]}\PY{p}{,}\PY{n}{SillyOracle}\PY{p}{)}\PY{p}{)}
\end{Verbatim}
\end{tcolorbox}

            \begin{tcolorbox}[breakable, boxrule=.5pt, size=fbox, pad at break*=1mm, opacityfill=0]
\prompt{Out}{outcolor}{30}{\hspace{3.5pt}}
\begin{Verbatim}[commandchars=\\\{\}]
'>=0.7\&<=0.8'
\end{Verbatim}
\end{tcolorbox}
        
    The oracle is not defined (returns \texttt{None}) and the result is the
unconditional probability.

    \begin{tcolorbox}[breakable, size=fbox, boxrule=1pt, pad at break*=1mm,colback=cellbackground, colframe=cellborder]
\prompt{In}{incolor}{31}{\hspace{4pt}}
\begin{Verbatim}[commandchars=\\\{\}]
\PY{n}{show}\PY{p}{(}\PY{n}{T1}\PY{o}{.}\PY{n}{query}\PY{p}{(}\PY{l+s+s1}{\PYZsq{}}\PY{l+s+s1}{a}\PY{l+s+s1}{\PYZsq{}}\PY{p}{,}\PY{p}{[}\PY{n}{T2}\PY{p}{]}\PY{p}{,}\PY{n}{SillyOracle}\PY{p}{)}\PY{p}{)}
\end{Verbatim}
\end{tcolorbox}

            \begin{tcolorbox}[breakable, boxrule=.5pt, size=fbox, pad at break*=1mm, opacityfill=0]
\prompt{Out}{outcolor}{31}{\hspace{3.5pt}}
\begin{Verbatim}[commandchars=\\\{\}]
'0.6'
\end{Verbatim}
\end{tcolorbox}
        
    The oracle is defined but is ignored because of the subtyping condition.

    \begin{tcolorbox}[breakable, size=fbox, boxrule=1pt, pad at break*=1mm,colback=cellbackground, colframe=cellborder]
\prompt{In}{incolor}{32}{\hspace{4pt}}
\begin{Verbatim}[commandchars=\\\{\}]
\PY{n}{show}\PY{p}{(}\PY{n}{T1}\PY{o}{.}\PY{n}{query}\PY{p}{(}\PY{l+s+s1}{\PYZsq{}}\PY{l+s+s1}{a}\PY{l+s+s1}{\PYZsq{}}\PY{p}{,}\PY{p}{[}\PY{p}{(}\PY{l+s+s1}{\PYZsq{}}\PY{l+s+s1}{b}\PY{l+s+s1}{\PYZsq{}}\PY{p}{,}\PY{n}{T2}\PY{p}{)}\PY{p}{,}\PY{p}{(}\PY{l+s+s1}{\PYZsq{}}\PY{l+s+s1}{a}\PY{l+s+s1}{\PYZsq{}}\PY{p}{,}\PY{n}{T3}\PY{p}{)}\PY{p}{]}\PY{p}{,}\PY{n}{SillyOracle}\PY{p}{)}\PY{p}{)}
\end{Verbatim}
\end{tcolorbox}

            \begin{tcolorbox}[breakable, boxrule=.5pt, size=fbox, pad at break*=1mm, opacityfill=0]
\prompt{Out}{outcolor}{32}{\hspace{3.5pt}}
\begin{Verbatim}[commandchars=\\\{\}]
'1.0'
\end{Verbatim}
\end{tcolorbox}
        
    The probability that \(a\) is to the left of \(b\) is the same as the
probability the \(b\) is to the right of \(a\).

    \begin{tcolorbox}[breakable, size=fbox, boxrule=1pt, pad at break*=1mm,colback=cellbackground, colframe=cellborder]
\prompt{In}{incolor}{33}{\hspace{4pt}}
\begin{Verbatim}[commandchars=\\\{\}]
\PY{n}{Ind} \PY{o}{=} \PY{n}{BType}\PY{p}{(}\PY{l+s+s1}{\PYZsq{}}\PY{l+s+s1}{Ind}\PY{l+s+s1}{\PYZsq{}}\PY{p}{)}
\PY{n}{Ind}\PY{o}{.}\PY{n}{judge}\PY{p}{(}\PY{l+s+s1}{\PYZsq{}}\PY{l+s+s1}{a}\PY{l+s+s1}{\PYZsq{}}\PY{p}{)}
\PY{n}{Ind}\PY{o}{.}\PY{n}{judge}\PY{p}{(}\PY{l+s+s1}{\PYZsq{}}\PY{l+s+s1}{b}\PY{l+s+s1}{\PYZsq{}}\PY{p}{)}
\PY{n}{left} \PY{o}{=} \PY{n}{Pred}\PY{p}{(}\PY{l+s+s1}{\PYZsq{}}\PY{l+s+s1}{left}\PY{l+s+s1}{\PYZsq{}}\PY{p}{,}\PY{p}{[}\PY{n}{Ind}\PY{p}{,}\PY{n}{Ind}\PY{p}{]}\PY{p}{)}
\PY{n}{right} \PY{o}{=} \PY{n}{Pred}\PY{p}{(}\PY{l+s+s1}{\PYZsq{}}\PY{l+s+s1}{right}\PY{l+s+s1}{\PYZsq{}}\PY{p}{,}\PY{p}{[}\PY{n}{Ind}\PY{p}{,}\PY{n}{Ind}\PY{p}{]}\PY{p}{)}
\PY{n}{left}\PY{o}{.}\PY{n}{learn\PYZus{}witness\PYZus{}fun}\PY{p}{(}\PY{k}{lambda} \PY{n}{args}\PY{p}{:} \PY{n}{PType}\PY{p}{(}\PY{n}{right}\PY{p}{,}\PY{p}{[}\PY{n}{args}\PY{p}{[}\PY{l+m+mi}{1}\PY{p}{]}\PY{p}{,}\PY{n}{args}\PY{p}{[}\PY{l+m+mi}{0}\PY{p}{]}\PY{p}{]}\PY{p}{)}\PY{p}{)}
\PY{n}{right}\PY{o}{.}\PY{n}{learn\PYZus{}witness\PYZus{}fun}\PY{p}{(}\PY{k}{lambda} \PY{n}{args}\PY{p}{:} \PY{n}{PType}\PY{p}{(}\PY{n}{left}\PY{p}{,}\PY{p}{[}\PY{n}{args}\PY{p}{[}\PY{l+m+mi}{1}\PY{p}{]}\PY{p}{,}\PY{n}{args}\PY{p}{[}\PY{l+m+mi}{0}\PY{p}{]}\PY{p}{]}\PY{p}{)}\PY{p}{)}
\PY{n}{left\PYZus{}a\PYZus{}b} \PY{o}{=} \PY{n}{PType}\PY{p}{(}\PY{n}{left}\PY{p}{,}\PY{p}{[}\PY{l+s+s1}{\PYZsq{}}\PY{l+s+s1}{a}\PY{l+s+s1}{\PYZsq{}}\PY{p}{,}\PY{l+s+s1}{\PYZsq{}}\PY{l+s+s1}{b}\PY{l+s+s1}{\PYZsq{}}\PY{p}{]}\PY{p}{)}
\PY{n}{right\PYZus{}b\PYZus{}a} \PY{o}{=} \PY{n}{PType}\PY{p}{(}\PY{n}{right}\PY{p}{,}\PY{p}{[}\PY{l+s+s1}{\PYZsq{}}\PY{l+s+s1}{b}\PY{l+s+s1}{\PYZsq{}}\PY{p}{,}\PY{l+s+s1}{\PYZsq{}}\PY{l+s+s1}{a}\PY{l+s+s1}{\PYZsq{}}\PY{p}{]}\PY{p}{)}
\PY{n}{M} \PY{o}{=} \PY{n}{Possibility}\PY{p}{(}\PY{l+s+s1}{\PYZsq{}}\PY{l+s+s1}{M}\PY{l+s+s1}{\PYZsq{}}\PY{p}{)}
\PY{n}{right\PYZus{}b\PYZus{}a}\PY{o}{.}\PY{n}{in\PYZus{}poss}\PY{p}{(}\PY{n}{M}\PY{p}{)}\PY{o}{.}\PY{n}{judge}\PY{p}{(}\PY{l+s+s1}{\PYZsq{}}\PY{l+s+s1}{s1}\PY{l+s+s1}{\PYZsq{}}\PY{p}{,}\PY{o}{.}\PY{l+m+mi}{6}\PY{p}{)}
\PY{n}{left\PYZus{}a\PYZus{}b}\PY{o}{.}\PY{n}{in\PYZus{}poss}\PY{p}{(}\PY{n}{M}\PY{p}{)}\PY{o}{.}\PY{n}{judge}\PY{p}{(}\PY{l+s+s1}{\PYZsq{}}\PY{l+s+s1}{s2}\PY{l+s+s1}{\PYZsq{}}\PY{p}{,}\PY{o}{.}\PY{l+m+mi}{7}\PY{p}{)}
\PY{n+nb}{print}\PY{p}{(}\PY{n}{show}\PY{p}{(}\PY{n}{M}\PY{p}{)}\PY{p}{)}
\PY{n+nb}{print}\PY{p}{(}\PY{n}{show}\PY{p}{(}\PY{n}{left\PYZus{}a\PYZus{}b}\PY{o}{.}\PY{n}{in\PYZus{}poss}\PY{p}{(}\PY{n}{M}\PY{p}{)}\PY{o}{.}\PY{n}{query}\PY{p}{(}\PY{l+s+s1}{\PYZsq{}}\PY{l+s+s1}{s1}\PY{l+s+s1}{\PYZsq{}}\PY{p}{)}\PY{p}{)}\PY{p}{)}
\PY{n+nb}{print}\PY{p}{(}\PY{n}{show}\PY{p}{(}\PY{n}{right\PYZus{}b\PYZus{}a}\PY{o}{.}\PY{n}{in\PYZus{}poss}\PY{p}{(}\PY{n}{M}\PY{p}{)}\PY{o}{.}\PY{n}{query}\PY{p}{(}\PY{l+s+s1}{\PYZsq{}}\PY{l+s+s1}{s2}\PY{l+s+s1}{\PYZsq{}}\PY{p}{)}\PY{p}{)}\PY{p}{)}
\PY{n+nb}{print}\PY{p}{(}\PY{n}{show}\PY{p}{(}\PY{n}{M}\PY{p}{)}\PY{p}{)}
\end{Verbatim}
\end{tcolorbox}

    \begin{Verbatim}[commandchars=\\\{\}]

M:
\_\_\_\_\_\_\_\_\_\_\_\_\_\_\_\_\_\_\_\_\_\_\_\_\_\_\_\_\_\_\_\_\_\_\_\_\_\_\_\_\_\_\_\_\_
right(b, a): [(s1, 0.6)]
left(a, b): [(s2, 0.7)]
\_\_\_\_\_\_\_\_\_\_\_\_\_\_\_\_\_\_\_\_\_\_\_\_\_\_\_\_\_\_\_\_\_\_\_\_\_\_\_\_\_\_\_\_\_

0.6
0.7

M:
\_\_\_\_\_\_\_\_\_\_\_\_\_\_\_\_\_\_\_\_\_\_\_\_\_\_\_\_\_\_\_\_\_\_\_\_\_\_\_\_\_\_\_\_\_
right(b, a): [(s1, 0.6), (s2, 0.7)]
left(a, b): [(s2, 0.7), (s1, 0.6)]
\_\_\_\_\_\_\_\_\_\_\_\_\_\_\_\_\_\_\_\_\_\_\_\_\_\_\_\_\_\_\_\_\_\_\_\_\_\_\_\_\_\_\_\_\_

\end{Verbatim}

    What happens if we revise our original judgement? This shows the
downside of caching the result of an inference in a dynamic environment
where a premise for the inference has changed since we last checked.
Below we rejudge \texttt{left(a,b)} to have probability \texttt{.3} in
\texttt{s2}. However, when we query \texttt{right(b,a)} in \texttt{s2},
we still have the old value we found with the earlier judgement for
\texttt{left(a,b)}.

    \begin{tcolorbox}[breakable, size=fbox, boxrule=1pt, pad at break*=1mm,colback=cellbackground, colframe=cellborder]
\prompt{In}{incolor}{34}{\hspace{4pt}}
\begin{Verbatim}[commandchars=\\\{\}]
\PY{n}{left\PYZus{}a\PYZus{}b}\PY{o}{.}\PY{n}{in\PYZus{}poss}\PY{p}{(}\PY{n}{M}\PY{p}{)}\PY{o}{.}\PY{n}{judge}\PY{p}{(}\PY{l+s+s1}{\PYZsq{}}\PY{l+s+s1}{s2}\PY{l+s+s1}{\PYZsq{}}\PY{p}{,}\PY{o}{.}\PY{l+m+mi}{3}\PY{p}{)}
\PY{n+nb}{print}\PY{p}{(}\PY{n}{show}\PY{p}{(}\PY{n}{M}\PY{p}{)}\PY{p}{)}
\PY{n+nb}{print}\PY{p}{(}\PY{n}{show}\PY{p}{(}\PY{n}{right\PYZus{}b\PYZus{}a}\PY{o}{.}\PY{n}{in\PYZus{}poss}\PY{p}{(}\PY{n}{M}\PY{p}{)}\PY{o}{.}\PY{n}{query}\PY{p}{(}\PY{l+s+s1}{\PYZsq{}}\PY{l+s+s1}{s2}\PY{l+s+s1}{\PYZsq{}}\PY{p}{)}\PY{p}{)}\PY{p}{)}
\PY{n+nb}{print}\PY{p}{(}\PY{n}{show}\PY{p}{(}\PY{n}{M}\PY{p}{)}\PY{p}{)}
\end{Verbatim}
\end{tcolorbox}

    \begin{Verbatim}[commandchars=\\\{\}]

M:
\_\_\_\_\_\_\_\_\_\_\_\_\_\_\_\_\_\_\_\_\_\_\_\_\_\_\_\_\_\_\_\_\_\_\_\_\_\_\_\_\_\_\_\_\_
right(b, a): [(s1, 0.6), (s2, 0.7)]
left(a, b): [(s2, 0.3), (s1, 0.6)]
\_\_\_\_\_\_\_\_\_\_\_\_\_\_\_\_\_\_\_\_\_\_\_\_\_\_\_\_\_\_\_\_\_\_\_\_\_\_\_\_\_\_\_\_\_

0.7

M:
\_\_\_\_\_\_\_\_\_\_\_\_\_\_\_\_\_\_\_\_\_\_\_\_\_\_\_\_\_\_\_\_\_\_\_\_\_\_\_\_\_\_\_\_\_
right(b, a): [(s1, 0.6), (s2, 0.7)]
left(a, b): [(s2, 0.3), (s1, 0.6)]
\_\_\_\_\_\_\_\_\_\_\_\_\_\_\_\_\_\_\_\_\_\_\_\_\_\_\_\_\_\_\_\_\_\_\_\_\_\_\_\_\_\_\_\_\_

\end{Verbatim}

    This is clearly the wrong result for the new circumstances. A way around
this is to \texttt{forget()} the probability for \texttt{s2} before
calling \texttt{query()}. Of course, knowing what needs to be forgotten
is a delicate problem related to work on belief revision. A good
strategy is perhaps to always forget before querying when there is any
possibility that something relevant might have changed.
\texttt{forget()} returns the old probability it found just in case you
want to save it and reinstate it later if you are unable to compute a
new probability.

    \begin{tcolorbox}[breakable, size=fbox, boxrule=1pt, pad at break*=1mm,colback=cellbackground, colframe=cellborder]
\prompt{In}{incolor}{35}{\hspace{4pt}}
\begin{Verbatim}[commandchars=\\\{\}]
\PY{n+nb}{print}\PY{p}{(}\PY{n}{show}\PY{p}{(}\PY{n}{right\PYZus{}b\PYZus{}a}\PY{o}{.}\PY{n}{in\PYZus{}poss}\PY{p}{(}\PY{n}{M}\PY{p}{)}\PY{o}{.}\PY{n}{forget}\PY{p}{(}\PY{l+s+s1}{\PYZsq{}}\PY{l+s+s1}{s2}\PY{l+s+s1}{\PYZsq{}}\PY{p}{)}\PY{p}{)}\PY{p}{)}
\PY{n+nb}{print}\PY{p}{(}\PY{n}{show}\PY{p}{(}\PY{n}{M}\PY{p}{)}\PY{p}{)}
\PY{n+nb}{print}\PY{p}{(}\PY{n}{show}\PY{p}{(}\PY{n}{right\PYZus{}b\PYZus{}a}\PY{o}{.}\PY{n}{in\PYZus{}poss}\PY{p}{(}\PY{n}{M}\PY{p}{)}\PY{o}{.}\PY{n}{query}\PY{p}{(}\PY{l+s+s1}{\PYZsq{}}\PY{l+s+s1}{s2}\PY{l+s+s1}{\PYZsq{}}\PY{p}{)}\PY{p}{)}\PY{p}{)}
\PY{n+nb}{print}\PY{p}{(}\PY{n}{show}\PY{p}{(}\PY{n}{M}\PY{p}{)}\PY{p}{)}
\end{Verbatim}
\end{tcolorbox}

    \begin{Verbatim}[commandchars=\\\{\}]
0.7

M:
\_\_\_\_\_\_\_\_\_\_\_\_\_\_\_\_\_\_\_\_\_\_\_\_\_\_\_\_\_\_\_\_\_\_\_\_\_\_\_\_\_\_\_\_\_
right(b, a): [(s1, 0.6)]
left(a, b): [(s2, 0.3), (s1, 0.6)]
\_\_\_\_\_\_\_\_\_\_\_\_\_\_\_\_\_\_\_\_\_\_\_\_\_\_\_\_\_\_\_\_\_\_\_\_\_\_\_\_\_\_\_\_\_

0.3

M:
\_\_\_\_\_\_\_\_\_\_\_\_\_\_\_\_\_\_\_\_\_\_\_\_\_\_\_\_\_\_\_\_\_\_\_\_\_\_\_\_\_\_\_\_\_
right(b, a): [(s1, 0.6), (s2, 0.3)]
left(a, b): [(s2, 0.3), (s1, 0.6)]
\_\_\_\_\_\_\_\_\_\_\_\_\_\_\_\_\_\_\_\_\_\_\_\_\_\_\_\_\_\_\_\_\_\_\_\_\_\_\_\_\_\_\_\_\_

\end{Verbatim}

    \hypertarget{non-specific-querying}{%
\subsection{Non-specific querying}\label{non-specific-querying}}

    The query \texttt{T.query\_nonspec()} asks for the probability that
there is something of type \texttt{T}, \(p(T)\) in the notation of
probabilistic TTR. If something has been judged to be of type \texttt{T}
with probability 1, then \texttt{T.query\_nonspec()} returns probability
1.

    \begin{tcolorbox}[breakable, size=fbox, boxrule=1pt, pad at break*=1mm,colback=cellbackground, colframe=cellborder]
\prompt{In}{incolor}{36}{\hspace{4pt}}
\begin{Verbatim}[commandchars=\\\{\}]
\PY{n}{show}\PY{p}{(}\PY{n}{Ind}\PY{o}{.}\PY{n}{query\PYZus{}nonspec}\PY{p}{(}\PY{p}{)}\PY{p}{)}
\end{Verbatim}
\end{tcolorbox}

            \begin{tcolorbox}[breakable, boxrule=.5pt, size=fbox, pad at break*=1mm, opacityfill=0]
\prompt{Out}{outcolor}{36}{\hspace{3.5pt}}
\begin{Verbatim}[commandchars=\\\{\}]
'1.0'
\end{Verbatim}
\end{tcolorbox}
        
    Alternatively, if there is nothing in the witness cache but we have made
a non-specific judgement using \texttt{judge\_nonspec}, then the result
of that non-specific judgement is returned.

    \begin{tcolorbox}[breakable, size=fbox, boxrule=1pt, pad at break*=1mm,colback=cellbackground, colframe=cellborder]
\prompt{In}{incolor}{37}{\hspace{4pt}}
\begin{Verbatim}[commandchars=\\\{\}]
\PY{n}{show}\PY{p}{(}\PY{n}{T\PYZus{}new}\PY{o}{.}\PY{n}{query\PYZus{}nonspec}\PY{p}{(}\PY{p}{)}\PY{p}{)}
\end{Verbatim}
\end{tcolorbox}

            \begin{tcolorbox}[breakable, boxrule=.5pt, size=fbox, pad at break*=1mm, opacityfill=0]
\prompt{Out}{outcolor}{37}{\hspace{3.5pt}}
\begin{Verbatim}[commandchars=\\\{\}]
'>=0.3\&<=0.4'
\end{Verbatim}
\end{tcolorbox}
        
    If no non-specific judgement has been made, then what is returned is the
disjunctive probability of all the probabilities in the witness cache.

    \begin{tcolorbox}[breakable, size=fbox, boxrule=1pt, pad at break*=1mm,colback=cellbackground, colframe=cellborder]
\prompt{In}{incolor}{38}{\hspace{4pt}}
\begin{Verbatim}[commandchars=\\\{\}]
\PY{n+nb}{print}\PY{p}{(}\PY{n}{show}\PY{p}{(}\PY{n}{M}\PY{p}{)}\PY{p}{)}
\PY{n+nb}{print}\PY{p}{(}\PY{n}{show}\PY{p}{(}\PY{n}{right\PYZus{}b\PYZus{}a}\PY{o}{.}\PY{n}{in\PYZus{}poss}\PY{p}{(}\PY{n}{M}\PY{p}{)}\PY{o}{.}\PY{n}{query\PYZus{}nonspec}\PY{p}{(}\PY{p}{)}\PY{p}{)}\PY{p}{)}
\PY{n+nb}{print}\PY{p}{(}\PY{n}{show}\PY{p}{(}\PY{n}{left\PYZus{}a\PYZus{}b}\PY{o}{.}\PY{n}{in\PYZus{}poss}\PY{p}{(}\PY{n}{M}\PY{p}{)}\PY{o}{.}\PY{n}{query\PYZus{}nonspec}\PY{p}{(}\PY{p}{)}\PY{p}{)}\PY{p}{)}
\PY{n+nb}{print}\PY{p}{(}\PY{n}{show}\PY{p}{(}\PY{n}{M}\PY{p}{)}\PY{p}{)}
\end{Verbatim}
\end{tcolorbox}

    \begin{Verbatim}[commandchars=\\\{\}]

M:
\_\_\_\_\_\_\_\_\_\_\_\_\_\_\_\_\_\_\_\_\_\_\_\_\_\_\_\_\_\_\_\_\_\_\_\_\_\_\_\_\_\_\_\_\_
right(b, a): [(s1, 0.6), (s2, 0.3)]
left(a, b): [(s2, 0.3), (s1, 0.6)]
\_\_\_\_\_\_\_\_\_\_\_\_\_\_\_\_\_\_\_\_\_\_\_\_\_\_\_\_\_\_\_\_\_\_\_\_\_\_\_\_\_\_\_\_\_

0.72
0.72

M:
\_\_\_\_\_\_\_\_\_\_\_\_\_\_\_\_\_\_\_\_\_\_\_\_\_\_\_\_\_\_\_\_\_\_\_\_\_\_\_\_\_\_\_\_\_
right(b, a): [(s1, 0.6), (s2, 0.3)]
left(a, b): [(s2, 0.3), (s1, 0.6)]
\_\_\_\_\_\_\_\_\_\_\_\_\_\_\_\_\_\_\_\_\_\_\_\_\_\_\_\_\_\_\_\_\_\_\_\_\_\_\_\_\_\_\_\_\_

\end{Verbatim}

    If a non-specific judgement has been made and there is a non-empty
witness cache, then what is returned is the maximum of the non-specific
judgement and the disjunctive probability of the probabilities in the
witness cache. Suppose, for example, that I make a judgement that there
is .5 probability that there is something of type \texttt{Ind}, contrary
to what is represented in the witness cache which shows objects with
probability 1. Then the result from the witness cache takes precedence.
If on the other hand I have made a judgement that there is certain
likelihood of something being of a certain type and this exceeds the
evidence represented in the witness cache, then the non-specific
judgement is returned.

    \begin{tcolorbox}[breakable, size=fbox, boxrule=1pt, pad at break*=1mm,colback=cellbackground, colframe=cellborder]
\prompt{In}{incolor}{39}{\hspace{4pt}}
\begin{Verbatim}[commandchars=\\\{\}]
\PY{n}{Ind}\PY{o}{.}\PY{n}{judge\PYZus{}nonspec}\PY{p}{(}\PY{o}{.}\PY{l+m+mi}{5}\PY{p}{)}
\PY{n+nb}{print}\PY{p}{(}\PY{n}{show}\PY{p}{(}\PY{n}{Ind}\PY{o}{.}\PY{n}{query\PYZus{}nonspec}\PY{p}{(}\PY{p}{)}\PY{p}{)}\PY{p}{)}
\PY{n}{T\PYZus{}new}\PY{o}{.}\PY{n}{judge}\PY{p}{(}\PY{l+s+s1}{\PYZsq{}}\PY{l+s+s1}{a}\PY{l+s+s1}{\PYZsq{}}\PY{p}{,}\PY{o}{.}\PY{l+m+mi}{1}\PY{p}{)}
\PY{n+nb}{print}\PY{p}{(}\PY{n}{show}\PY{p}{(}\PY{n}{T\PYZus{}new}\PY{o}{.}\PY{n}{query\PYZus{}nonspec}\PY{p}{(}\PY{p}{)}\PY{p}{)}\PY{p}{)}
\end{Verbatim}
\end{tcolorbox}

    \begin{Verbatim}[commandchars=\\\{\}]
1.0
>=0.3\&<=0.4
\end{Verbatim}

    Non-specific queries can also be conditional. Here we show the
probability that there is some situation in the type \texttt{right(b,a)}
(i.e.~that it is true that \texttt{b} is to the right of \texttt{a})
given that \texttt{s3} is a witness of \texttt{left(a,b)}.

    \begin{tcolorbox}[breakable, size=fbox, boxrule=1pt, pad at break*=1mm,colback=cellbackground, colframe=cellborder]
\prompt{In}{incolor}{40}{\hspace{4pt}}
\begin{Verbatim}[commandchars=\\\{\}]
\PY{n+nb}{print}\PY{p}{(}\PY{n}{show}\PY{p}{(}\PY{n}{right\PYZus{}b\PYZus{}a}\PY{o}{.}\PY{n}{in\PYZus{}poss}\PY{p}{(}\PY{n}{M}\PY{p}{)}\PY{o}{.}\PY{n}{query\PYZus{}nonspec}\PY{p}{(}\PY{p}{[}\PY{p}{(}\PY{l+s+s1}{\PYZsq{}}\PY{l+s+s1}{s3}\PY{l+s+s1}{\PYZsq{}}\PY{p}{,}\PY{n}{left\PYZus{}a\PYZus{}b}\PY{o}{.}\PY{n}{in\PYZus{}poss}\PY{p}{(}\PY{n}{M}\PY{p}{)}\PY{p}{)}\PY{p}{]}\PY{p}{)}\PY{p}{)}\PY{p}{)}
\PY{n+nb}{print}\PY{p}{(}\PY{n}{show}\PY{p}{(}\PY{n}{M}\PY{p}{)}\PY{p}{)}
\end{Verbatim}
\end{tcolorbox}

    \begin{Verbatim}[commandchars=\\\{\}]
1.0

M:
\_\_\_\_\_\_\_\_\_\_\_\_\_\_\_\_\_\_\_\_\_\_\_\_\_\_\_\_\_\_\_\_\_\_\_\_\_\_\_\_\_\_\_\_\_
right(b, a): [(s1, 0.6), (s2, 0.3)]
left(a, b): [(s2, 0.3), (s1, 0.6)]
\_\_\_\_\_\_\_\_\_\_\_\_\_\_\_\_\_\_\_\_\_\_\_\_\_\_\_\_\_\_\_\_\_\_\_\_\_\_\_\_\_\_\_\_\_

\end{Verbatim}

    Below we show the probability that there is some witness for the type
\texttt{right(b,a)}, given that there is some witness for the type
\texttt{left(a,b)}.

    \begin{tcolorbox}[breakable, size=fbox, boxrule=1pt, pad at break*=1mm,colback=cellbackground, colframe=cellborder]
\prompt{In}{incolor}{41}{\hspace{4pt}}
\begin{Verbatim}[commandchars=\\\{\}]
\PY{n+nb}{print}\PY{p}{(}\PY{n}{show}\PY{p}{(}\PY{n}{right\PYZus{}b\PYZus{}a}\PY{o}{.}\PY{n}{in\PYZus{}poss}\PY{p}{(}\PY{n}{M}\PY{p}{)}\PY{o}{.}\PY{n}{query\PYZus{}nonspec}\PY{p}{(}\PY{p}{[}\PY{n}{left\PYZus{}a\PYZus{}b}\PY{o}{.}\PY{n}{in\PYZus{}poss}\PY{p}{(}\PY{n}{M}\PY{p}{)}\PY{p}{]}\PY{p}{)}\PY{p}{)}\PY{p}{)}
\PY{n+nb}{print}\PY{p}{(}\PY{n}{show}\PY{p}{(}\PY{n}{M}\PY{p}{)}\PY{p}{)}
\end{Verbatim}
\end{tcolorbox}

    \begin{Verbatim}[commandchars=\\\{\}]
1.0

M:
\_\_\_\_\_\_\_\_\_\_\_\_\_\_\_\_\_\_\_\_\_\_\_\_\_\_\_\_\_\_\_\_\_\_\_\_\_\_\_\_\_\_\_\_\_
right(b, a): [(s1, 0.6), (s2, 0.3)]
left(a, b): [(s2, 0.3), (s1, 0.6)]
\_\_\_\_\_\_\_\_\_\_\_\_\_\_\_\_\_\_\_\_\_\_\_\_\_\_\_\_\_\_\_\_\_\_\_\_\_\_\_\_\_\_\_\_\_

\end{Verbatim}

    If the type in the conditions is not a subtype of the type being queried
(in the case below because it refers to a different model), then what is
returned is the unconditional probability, unless we provide a relevant
oracle.

    \begin{tcolorbox}[breakable, size=fbox, boxrule=1pt, pad at break*=1mm,colback=cellbackground, colframe=cellborder]
\prompt{In}{incolor}{42}{\hspace{4pt}}
\begin{Verbatim}[commandchars=\\\{\}]
\PY{n}{M1} \PY{o}{=} \PY{n}{Possibility}\PY{p}{(}\PY{l+s+s1}{\PYZsq{}}\PY{l+s+s1}{M1}\PY{l+s+s1}{\PYZsq{}}\PY{p}{)}
\PY{n+nb}{print}\PY{p}{(}\PY{n}{show}\PY{p}{(}\PY{n}{right\PYZus{}b\PYZus{}a}\PY{o}{.}\PY{n}{in\PYZus{}poss}\PY{p}{(}\PY{n}{M}\PY{p}{)}\PY{o}{.}\PY{n}{query\PYZus{}nonspec}\PY{p}{(}\PY{p}{[}\PY{n}{left\PYZus{}a\PYZus{}b}\PY{o}{.}\PY{n}{in\PYZus{}poss}\PY{p}{(}\PY{n}{M1}\PY{p}{)}\PY{p}{]}\PY{p}{)}\PY{p}{)}\PY{p}{)}
\end{Verbatim}
\end{tcolorbox}

    \begin{Verbatim}[commandchars=\\\{\}]
0.72
\end{Verbatim}

    \hypertarget{meet-types}{%
\subsection{Meet types}\label{meet-types}}

    If we judge that the probability of \texttt{a} being of type
\texttt{MeetType(T1,T2)} is \texttt{1}, then we also judge the
probability of \texttt{a} being of \texttt{T1} and the probability of
\texttt{a} being of \texttt{T2} to be \texttt{1}.

    \begin{tcolorbox}[breakable, size=fbox, boxrule=1pt, pad at break*=1mm,colback=cellbackground, colframe=cellborder]
\prompt{In}{incolor}{43}{\hspace{4pt}}
\begin{Verbatim}[commandchars=\\\{\}]
\PY{n}{Tleft} \PY{o}{=} \PY{n}{Type}\PY{p}{(}\PY{p}{)}
\PY{n}{Tright} \PY{o}{=} \PY{n}{Type}\PY{p}{(}\PY{p}{)}
\PY{n}{Tm} \PY{o}{=} \PY{n}{MeetType}\PY{p}{(}\PY{n}{Tleft}\PY{p}{,}\PY{n}{Tright}\PY{p}{)}
\PY{n}{Tm}\PY{o}{.}\PY{n}{judge}\PY{p}{(}\PY{l+s+s1}{\PYZsq{}}\PY{l+s+s1}{a}\PY{l+s+s1}{\PYZsq{}}\PY{p}{)}
\PY{n+nb}{print}\PY{p}{(}\PY{n}{show}\PY{p}{(}\PY{n}{Tleft}\PY{o}{.}\PY{n}{query}\PY{p}{(}\PY{l+s+s1}{\PYZsq{}}\PY{l+s+s1}{a}\PY{l+s+s1}{\PYZsq{}}\PY{p}{)}\PY{p}{)}\PY{p}{)}
\PY{n+nb}{print}\PY{p}{(}\PY{n}{show}\PY{p}{(}\PY{n}{Tright}\PY{o}{.}\PY{n}{query}\PY{p}{(}\PY{l+s+s1}{\PYZsq{}}\PY{l+s+s1}{a}\PY{l+s+s1}{\PYZsq{}}\PY{p}{)}\PY{p}{)}\PY{p}{)}
\end{Verbatim}
\end{tcolorbox}

    \begin{Verbatim}[commandchars=\\\{\}]
1.0
1.0
\end{Verbatim}

    Otherwise, we do not currently draw any conclusions about the
probabilities for the component types.

    \begin{tcolorbox}[breakable, size=fbox, boxrule=1pt, pad at break*=1mm,colback=cellbackground, colframe=cellborder]
\prompt{In}{incolor}{44}{\hspace{4pt}}
\begin{Verbatim}[commandchars=\\\{\}]
\PY{n}{Tleft1} \PY{o}{=} \PY{n}{Type}\PY{p}{(}\PY{p}{)}
\PY{n}{Tright1} \PY{o}{=} \PY{n}{Type}\PY{p}{(}\PY{p}{)}
\PY{n}{Tm1} \PY{o}{=} \PY{n}{MeetType}\PY{p}{(}\PY{n}{Tleft1}\PY{p}{,}\PY{n}{Tright1}\PY{p}{)}
\PY{n}{Tm1}\PY{o}{.}\PY{n}{judge}\PY{p}{(}\PY{l+s+s1}{\PYZsq{}}\PY{l+s+s1}{a}\PY{l+s+s1}{\PYZsq{}}\PY{p}{,}\PY{o}{.}\PY{l+m+mi}{6}\PY{p}{,}\PY{o}{.}\PY{l+m+mi}{8}\PY{p}{)}
\PY{n+nb}{print}\PY{p}{(}\PY{n}{show}\PY{p}{(}\PY{n}{Tleft1}\PY{o}{.}\PY{n}{query}\PY{p}{(}\PY{l+s+s1}{\PYZsq{}}\PY{l+s+s1}{a}\PY{l+s+s1}{\PYZsq{}}\PY{p}{)}\PY{p}{)}\PY{p}{)}
\PY{n+nb}{print}\PY{p}{(}\PY{n}{show}\PY{p}{(}\PY{n}{Tright1}\PY{o}{.}\PY{n}{query}\PY{p}{(}\PY{l+s+s1}{\PYZsq{}}\PY{l+s+s1}{a}\PY{l+s+s1}{\PYZsq{}}\PY{p}{)}\PY{p}{)}\PY{p}{)}
\PY{n+nb}{print}\PY{p}{(}\PY{n}{show}\PY{p}{(}\PY{n}{Tm1}\PY{o}{.}\PY{n}{query}\PY{p}{(}\PY{l+s+s1}{\PYZsq{}}\PY{l+s+s1}{a}\PY{l+s+s1}{\PYZsq{}}\PY{p}{)}\PY{p}{)}\PY{p}{)}
\end{Verbatim}
\end{tcolorbox}

    \begin{Verbatim}[commandchars=\\\{\}]
<=1.0
<=1.0
>=0.6\&<=0.8
\end{Verbatim}

    The user may wish to decide that the \texttt{judge} method is not to be
used with meet types, only \texttt{query}, that is, that the
\texttt{judge} method is restricted to basic types. However, making
judgements about join types may be useful. See below.

    Similar remarks hold for non-specific judgements.

    \begin{tcolorbox}[breakable, size=fbox, boxrule=1pt, pad at break*=1mm,colback=cellbackground, colframe=cellborder]
\prompt{In}{incolor}{45}{\hspace{4pt}}
\begin{Verbatim}[commandchars=\\\{\}]
\PY{n}{Tleft2} \PY{o}{=} \PY{n}{Type}\PY{p}{(}\PY{p}{)}
\PY{n}{Tright2} \PY{o}{=} \PY{n}{Type}\PY{p}{(}\PY{p}{)}
\PY{n}{Tm2} \PY{o}{=} \PY{n}{MeetType}\PY{p}{(}\PY{n}{Tleft2}\PY{p}{,}\PY{n}{Tright2}\PY{p}{)}
\PY{n}{Tm2}\PY{o}{.}\PY{n}{judge\PYZus{}nonspec}\PY{p}{(}\PY{p}{)}
\PY{n+nb}{print}\PY{p}{(}\PY{n}{show}\PY{p}{(}\PY{n}{Tleft2}\PY{o}{.}\PY{n}{query\PYZus{}nonspec}\PY{p}{(}\PY{p}{)}\PY{p}{)}\PY{p}{)}
\PY{n+nb}{print}\PY{p}{(}\PY{n}{show}\PY{p}{(}\PY{n}{Tright2}\PY{o}{.}\PY{n}{query\PYZus{}nonspec}\PY{p}{(}\PY{p}{)}\PY{p}{)}\PY{p}{)}
\PY{n}{Tleft3} \PY{o}{=} \PY{n}{Type}\PY{p}{(}\PY{p}{)}
\PY{n}{Tright3} \PY{o}{=} \PY{n}{Type}\PY{p}{(}\PY{p}{)}
\PY{n}{Tm3} \PY{o}{=} \PY{n}{MeetType}\PY{p}{(}\PY{n}{Tleft3}\PY{p}{,}\PY{n}{Tright3}\PY{p}{)}
\PY{n}{Tm3}\PY{o}{.}\PY{n}{judge\PYZus{}nonspec}\PY{p}{(}\PY{o}{.}\PY{l+m+mi}{6}\PY{p}{,}\PY{o}{.}\PY{l+m+mi}{8}\PY{p}{)}
\PY{n+nb}{print}\PY{p}{(}\PY{n}{show}\PY{p}{(}\PY{n}{Tleft3}\PY{o}{.}\PY{n}{query\PYZus{}nonspec}\PY{p}{(}\PY{p}{)}\PY{p}{)}\PY{p}{)}
\PY{n+nb}{print}\PY{p}{(}\PY{n}{show}\PY{p}{(}\PY{n}{Tright3}\PY{o}{.}\PY{n}{query\PYZus{}nonspec}\PY{p}{(}\PY{p}{)}\PY{p}{)}\PY{p}{)}
\PY{n+nb}{print}\PY{p}{(}\PY{n}{show}\PY{p}{(}\PY{n}{Tm3}\PY{o}{.}\PY{n}{query\PYZus{}nonspec}\PY{p}{(}\PY{p}{)}\PY{p}{)}\PY{p}{)}
\end{Verbatim}
\end{tcolorbox}

    \begin{Verbatim}[commandchars=\\\{\}]
1.0
1.0
<=1.0
<=1.0
>=0.6\&<=0.8
\end{Verbatim}

    If an object is not in the witness cache of a meet type then the
conjunctive probability of the values returned for the two components is
returned.

    \begin{tcolorbox}[breakable, size=fbox, boxrule=1pt, pad at break*=1mm,colback=cellbackground, colframe=cellborder]
\prompt{In}{incolor}{46}{\hspace{4pt}}
\begin{Verbatim}[commandchars=\\\{\}]
\PY{n}{Tleft3}\PY{o}{.}\PY{n}{judge}\PY{p}{(}\PY{l+s+s1}{\PYZsq{}}\PY{l+s+s1}{a}\PY{l+s+s1}{\PYZsq{}}\PY{p}{,}\PY{o}{.}\PY{l+m+mi}{6}\PY{p}{)}
\PY{n+nb}{print}\PY{p}{(}\PY{n}{show}\PY{p}{(}\PY{n}{Tright3}\PY{o}{.}\PY{n}{query}\PY{p}{(}\PY{l+s+s1}{\PYZsq{}}\PY{l+s+s1}{a}\PY{l+s+s1}{\PYZsq{}}\PY{p}{)}\PY{p}{)}\PY{p}{)}
\PY{n}{show}\PY{p}{(}\PY{n}{Tm3}\PY{o}{.}\PY{n}{query}\PY{p}{(}\PY{l+s+s1}{\PYZsq{}}\PY{l+s+s1}{a}\PY{l+s+s1}{\PYZsq{}}\PY{p}{)}\PY{p}{)}
\end{Verbatim}
\end{tcolorbox}

    \begin{Verbatim}[commandchars=\\\{\}]
<=1.0
\end{Verbatim}

            \begin{tcolorbox}[breakable, boxrule=.5pt, size=fbox, pad at break*=1mm, opacityfill=0]
\prompt{Out}{outcolor}{46}{\hspace{3.5pt}}
\begin{Verbatim}[commandchars=\\\{\}]
'<=0.6'
\end{Verbatim}
\end{tcolorbox}
        
    If an object is in the witness cache then the probability stored there
will be returned, even though there may be conflicting evidence in the
two components. In order to get the new value we need to
\texttt{forget()}

    \begin{tcolorbox}[breakable, size=fbox, boxrule=1pt, pad at break*=1mm,colback=cellbackground, colframe=cellborder]
\prompt{In}{incolor}{47}{\hspace{4pt}}
\begin{Verbatim}[commandchars=\\\{\}]
\PY{n}{Tright3}\PY{o}{.}\PY{n}{judge}\PY{p}{(}\PY{l+s+s1}{\PYZsq{}}\PY{l+s+s1}{a}\PY{l+s+s1}{\PYZsq{}}\PY{p}{,}\PY{o}{.}\PY{l+m+mi}{3}\PY{p}{)}
\PY{n+nb}{print}\PY{p}{(}\PY{n}{show}\PY{p}{(}\PY{n}{Tm3}\PY{o}{.}\PY{n}{query}\PY{p}{(}\PY{l+s+s1}{\PYZsq{}}\PY{l+s+s1}{a}\PY{l+s+s1}{\PYZsq{}}\PY{p}{)}\PY{p}{)}\PY{p}{)}
\PY{n}{Tm3}\PY{o}{.}\PY{n}{forget}\PY{p}{(}\PY{l+s+s1}{\PYZsq{}}\PY{l+s+s1}{a}\PY{l+s+s1}{\PYZsq{}}\PY{p}{)}
\PY{n+nb}{print}\PY{p}{(}\PY{n}{show}\PY{p}{(}\PY{n}{Tm3}\PY{o}{.}\PY{n}{query}\PY{p}{(}\PY{l+s+s1}{\PYZsq{}}\PY{l+s+s1}{a}\PY{l+s+s1}{\PYZsq{}}\PY{p}{)}\PY{p}{)}\PY{p}{)}
\end{Verbatim}
\end{tcolorbox}

    \begin{Verbatim}[commandchars=\\\{\}]
<=0.6
0.18
\end{Verbatim}

    The computation of conjunctive probability uses an adaptation of the
Kolmogorov formula for conjunction:
\(p(a:T_1\wedge T_2) = p(a:T_1)p(a:T_2\mid a:T_1)\), as given in Cooper
et al.~(2015). This involves a conditional probability and therefore in
the implementation and oracle argument creating a dependence between the
two types will make a difference to the outcome when querying a meet
type.

    \begin{tcolorbox}[breakable, size=fbox, boxrule=1pt, pad at break*=1mm,colback=cellbackground, colframe=cellborder]
\prompt{In}{incolor}{48}{\hspace{4pt}}
\begin{Verbatim}[commandchars=\\\{\}]
\PY{k}{def} \PY{n+nf}{Oracle1}\PY{p}{(}\PY{n}{a}\PY{p}{,}\PY{n}{T}\PY{p}{,}\PY{n}{c}\PY{p}{)}\PY{p}{:}
    \PY{k}{if} \PY{n}{a} \PY{o+ow}{is}\PY{l+s+s1}{\PYZsq{}}\PY{l+s+s1}{a}\PY{l+s+s1}{\PYZsq{}}\PY{o+ow}{and} \PY{n}{T} \PY{o+ow}{is} \PY{n}{Tright3} \PY{o+ow}{and} \PY{p}{(}\PY{l+s+s1}{\PYZsq{}}\PY{l+s+s1}{a}\PY{l+s+s1}{\PYZsq{}}\PY{p}{,}\PY{n}{Tleft3}\PY{p}{)} \PY{o+ow}{in} \PY{n}{c}\PY{p}{:}
        \PY{k}{return} \PY{n}{PConstraint}\PY{p}{(}\PY{o}{.}\PY{l+m+mi}{7}\PY{p}{,}\PY{o}{.}\PY{l+m+mi}{8}\PY{p}{)}
\PY{k}{def} \PY{n+nf}{Oracle2}\PY{p}{(}\PY{n}{a}\PY{p}{,}\PY{n}{T}\PY{p}{,}\PY{n}{c}\PY{p}{)}\PY{p}{:}
    \PY{k}{if} \PY{n}{a} \PY{o+ow}{is}\PY{l+s+s1}{\PYZsq{}}\PY{l+s+s1}{a}\PY{l+s+s1}{\PYZsq{}}\PY{o+ow}{and} \PY{n}{T} \PY{o+ow}{is} \PY{n}{Tright3} \PY{o+ow}{and} \PY{p}{(}\PY{l+s+s1}{\PYZsq{}}\PY{l+s+s1}{a}\PY{l+s+s1}{\PYZsq{}}\PY{p}{,}\PY{n}{Tleft3}\PY{p}{)} \PY{o+ow}{in} \PY{n}{c}\PY{p}{:}
        \PY{k}{return} \PY{n}{PConstraint}\PY{p}{(}\PY{l+m+mi}{0}\PY{p}{)}
\PY{n}{Tm3}\PY{o}{.}\PY{n}{forget}\PY{p}{(}\PY{l+s+s1}{\PYZsq{}}\PY{l+s+s1}{a}\PY{l+s+s1}{\PYZsq{}}\PY{p}{)}
\PY{n+nb}{print}\PY{p}{(}\PY{n}{show}\PY{p}{(}\PY{n}{Tm3}\PY{o}{.}\PY{n}{query}\PY{p}{(}\PY{l+s+s1}{\PYZsq{}}\PY{l+s+s1}{a}\PY{l+s+s1}{\PYZsq{}}\PY{p}{,}\PY{n}{oracle}\PY{o}{=}\PY{n}{Oracle1}\PY{p}{)}\PY{p}{)}\PY{p}{)}
\PY{n}{Tm3}\PY{o}{.}\PY{n}{forget}\PY{p}{(}\PY{l+s+s1}{\PYZsq{}}\PY{l+s+s1}{a}\PY{l+s+s1}{\PYZsq{}}\PY{p}{)}
\PY{n+nb}{print}\PY{p}{(}\PY{n}{show}\PY{p}{(}\PY{n}{Tm3}\PY{o}{.}\PY{n}{query}\PY{p}{(}\PY{l+s+s1}{\PYZsq{}}\PY{l+s+s1}{a}\PY{l+s+s1}{\PYZsq{}}\PY{p}{,}\PY{n}{oracle}\PY{o}{=}\PY{n}{Oracle2}\PY{p}{)}\PY{p}{)}\PY{p}{)}
\end{Verbatim}
\end{tcolorbox}

    \begin{Verbatim}[commandchars=\\\{\}]
>=0.42\&<=0.48
0.0
\end{Verbatim}

    Conditional probabilities can also be queried for meet types.

    \begin{tcolorbox}[breakable, size=fbox, boxrule=1pt, pad at break*=1mm,colback=cellbackground, colframe=cellborder]
\prompt{In}{incolor}{49}{\hspace{4pt}}
\begin{Verbatim}[commandchars=\\\{\}]
\PY{n}{Tm3}\PY{o}{.}\PY{n}{forget}\PY{p}{(}\PY{l+s+s1}{\PYZsq{}}\PY{l+s+s1}{a}\PY{l+s+s1}{\PYZsq{}}\PY{p}{)}
\PY{n+nb}{print}\PY{p}{(}\PY{n}{show}\PY{p}{(}\PY{n}{Tm3}\PY{o}{.}\PY{n}{query}\PY{p}{(}\PY{l+s+s1}{\PYZsq{}}\PY{l+s+s1}{a}\PY{l+s+s1}{\PYZsq{}}\PY{p}{,}\PY{p}{[}\PY{p}{(}\PY{l+s+s1}{\PYZsq{}}\PY{l+s+s1}{a}\PY{l+s+s1}{\PYZsq{}}\PY{p}{,}\PY{n}{Tleft3}\PY{p}{)}\PY{p}{,}\PY{p}{(}\PY{l+s+s1}{\PYZsq{}}\PY{l+s+s1}{a}\PY{l+s+s1}{\PYZsq{}}\PY{p}{,}\PY{n}{Tright3}\PY{p}{)}\PY{p}{]}\PY{p}{)}\PY{p}{)}\PY{p}{)}
\end{Verbatim}
\end{tcolorbox}

    \begin{Verbatim}[commandchars=\\\{\}]
1.0
\end{Verbatim}

    The above example shows the need for witness conditions which pass the
conditions and oracle arguments to components of the type. The witness
condition for meet types is, schematically,
\texttt{lambda\ a,c,oracle:\ ConjProb({[}(a,\textless{}left\textgreater{}),(a,\textless{}right\textgreater{}){]},c,oracle)}.
(As in non-probabilistic TTR, meet types and other ``logical types''
cannot learn new witness conditions.) Note that this witness condition
has three arguments so that it can pass the conditions, \texttt{c}, and
the oracle to the function \texttt{ConjProb} which computes the
conjunctive probability. Witness conditions in the \texttt{probttr}
implementation can have one to three arguments and will be applied to
the arguments provided to the \texttt{query} method from which they are
called: the object, \texttt{a}, being queried for one argument,
\texttt{a} and the conditions, \texttt{c}, for two arguments and
\texttt{a}, \texttt{c} and the oracle for three arguments.

    \hypertarget{join-types}{%
\subsection{Join types}\label{join-types}}

    Join types work like meet types \emph{mutatis mutandis}. Schematically,
the witness condition is
\texttt{lambda\ a,c,oracle:\ DisjProb({[}(a,\textless{}left\textgreater{}),(a,\textless{}right\textgreater{}){]},c,oracle)}
and no new witness conditions can be learnt. If we judge something to
have 0 probability of being of a join type, then we judge it to have 0
probability of being of the two component types.

    \begin{tcolorbox}[breakable, size=fbox, boxrule=1pt, pad at break*=1mm,colback=cellbackground, colframe=cellborder]
\prompt{In}{incolor}{50}{\hspace{4pt}}
\begin{Verbatim}[commandchars=\\\{\}]
\PY{n}{Tleftd} \PY{o}{=} \PY{n}{Type}\PY{p}{(}\PY{p}{)}
\PY{n}{Trightd} \PY{o}{=} \PY{n}{Type}\PY{p}{(}\PY{p}{)}
\PY{n}{Tmd} \PY{o}{=} \PY{n}{JoinType}\PY{p}{(}\PY{n}{Tleftd}\PY{p}{,}\PY{n}{Trightd}\PY{p}{)}
\PY{n}{Tmd}\PY{o}{.}\PY{n}{judge}\PY{p}{(}\PY{l+s+s1}{\PYZsq{}}\PY{l+s+s1}{a}\PY{l+s+s1}{\PYZsq{}}\PY{p}{,}\PY{l+m+mi}{0}\PY{p}{)}
\PY{n+nb}{print}\PY{p}{(}\PY{n}{show}\PY{p}{(}\PY{n}{Tleftd}\PY{o}{.}\PY{n}{query}\PY{p}{(}\PY{l+s+s1}{\PYZsq{}}\PY{l+s+s1}{a}\PY{l+s+s1}{\PYZsq{}}\PY{p}{)}\PY{p}{)}\PY{p}{)}
\PY{n+nb}{print}\PY{p}{(}\PY{n}{show}\PY{p}{(}\PY{n}{Trightd}\PY{o}{.}\PY{n}{query}\PY{p}{(}\PY{l+s+s1}{\PYZsq{}}\PY{l+s+s1}{a}\PY{l+s+s1}{\PYZsq{}}\PY{p}{)}\PY{p}{)}\PY{p}{)}
\end{Verbatim}
\end{tcolorbox}

    \begin{Verbatim}[commandchars=\\\{\}]
0.0
0.0
\end{Verbatim}

    Similarly for non-specific judgements.

    \begin{tcolorbox}[breakable, size=fbox, boxrule=1pt, pad at break*=1mm,colback=cellbackground, colframe=cellborder]
\prompt{In}{incolor}{51}{\hspace{4pt}}
\begin{Verbatim}[commandchars=\\\{\}]
\PY{n}{Tleft2d} \PY{o}{=} \PY{n}{Type}\PY{p}{(}\PY{p}{)}
\PY{n}{Tright2d} \PY{o}{=} \PY{n}{Type}\PY{p}{(}\PY{p}{)}
\PY{n}{Tm2d} \PY{o}{=} \PY{n}{JoinType}\PY{p}{(}\PY{n}{Tleft2d}\PY{p}{,}\PY{n}{Tright2d}\PY{p}{)}
\PY{n}{Tm2d}\PY{o}{.}\PY{n}{judge\PYZus{}nonspec}\PY{p}{(}\PY{l+m+mi}{0}\PY{p}{)}
\PY{n+nb}{print}\PY{p}{(}\PY{n}{show}\PY{p}{(}\PY{n}{Tleft2d}\PY{o}{.}\PY{n}{query\PYZus{}nonspec}\PY{p}{(}\PY{p}{)}\PY{p}{)}\PY{p}{)}
\PY{n+nb}{print}\PY{p}{(}\PY{n}{show}\PY{p}{(}\PY{n}{Tright2d}\PY{o}{.}\PY{n}{query\PYZus{}nonspec}\PY{p}{(}\PY{p}{)}\PY{p}{)}\PY{p}{)}
\PY{n}{Tleft3d} \PY{o}{=} \PY{n}{Type}\PY{p}{(}\PY{p}{)}
\PY{n}{Tright3d} \PY{o}{=} \PY{n}{Type}\PY{p}{(}\PY{p}{)}
\PY{n}{Tm3d} \PY{o}{=} \PY{n}{JoinType}\PY{p}{(}\PY{n}{Tleft3}\PY{p}{,}\PY{n}{Tright3}\PY{p}{)}
\PY{n}{Tm3d}\PY{o}{.}\PY{n}{judge\PYZus{}nonspec}\PY{p}{(}\PY{o}{.}\PY{l+m+mi}{6}\PY{p}{,}\PY{o}{.}\PY{l+m+mi}{8}\PY{p}{)}
\PY{n+nb}{print}\PY{p}{(}\PY{n}{show}\PY{p}{(}\PY{n}{Tleft3d}\PY{o}{.}\PY{n}{query\PYZus{}nonspec}\PY{p}{(}\PY{p}{)}\PY{p}{)}\PY{p}{)}
\PY{n+nb}{print}\PY{p}{(}\PY{n}{show}\PY{p}{(}\PY{n}{Tright3d}\PY{o}{.}\PY{n}{query\PYZus{}nonspec}\PY{p}{(}\PY{p}{)}\PY{p}{)}\PY{p}{)}
\PY{n+nb}{print}\PY{p}{(}\PY{n}{show}\PY{p}{(}\PY{n}{Tm3d}\PY{o}{.}\PY{n}{query\PYZus{}nonspec}\PY{p}{(}\PY{p}{)}\PY{p}{)}\PY{p}{)}
\end{Verbatim}
\end{tcolorbox}

    \begin{Verbatim}[commandchars=\\\{\}]
0.0
0.0
<=1.0
<=1.0
>=0.6\&<=0.8
\end{Verbatim}

    \hypertarget{record-types}{%
\subsection{Record types}\label{record-types}}

    The probability that a record, \(r\), is of a record type, \(T\), is the
conjunctive probability of the probabilities that the objects in the
fields of \(r\) are of the types in the correspondingly labelled types
of \(T\). If there is a label in \(T\) which is not in \(r\) then the
probability that \(r\) is of type \(T\) is 0 (also if \(r\) is not a
record at all then the probability is 0).

    A simple non-dependent record type.

    \begin{tcolorbox}[breakable, size=fbox, boxrule=1pt, pad at break*=1mm,colback=cellbackground, colframe=cellborder]
\prompt{In}{incolor}{52}{\hspace{4pt}}
\begin{Verbatim}[commandchars=\\\{\}]
\PY{n}{Tf1} \PY{o}{=} \PY{n}{Type}\PY{p}{(}\PY{p}{)}
\PY{n}{Tf2} \PY{o}{=} \PY{n}{Type}\PY{p}{(}\PY{p}{)}
\PY{n}{Tr1} \PY{o}{=} \PY{n}{RecType}\PY{p}{(}\PY{p}{\PYZob{}}\PY{l+s+s1}{\PYZsq{}}\PY{l+s+s1}{l1}\PY{l+s+s1}{\PYZsq{}}\PY{p}{:}\PY{n}{Tf1}\PY{p}{,}\PY{l+s+s1}{\PYZsq{}}\PY{l+s+s1}{l2}\PY{l+s+s1}{\PYZsq{}}\PY{p}{:}\PY{n}{Tf2}\PY{p}{\PYZcb{}}\PY{p}{)}
\PY{n}{Tf1}\PY{o}{.}\PY{n}{judge}\PY{p}{(}\PY{l+s+s1}{\PYZsq{}}\PY{l+s+s1}{a}\PY{l+s+s1}{\PYZsq{}}\PY{p}{,}\PY{o}{.}\PY{l+m+mi}{3}\PY{p}{)}
\PY{n}{Tf2}\PY{o}{.}\PY{n}{judge}\PY{p}{(}\PY{l+s+s1}{\PYZsq{}}\PY{l+s+s1}{b}\PY{l+s+s1}{\PYZsq{}}\PY{p}{,}\PY{o}{.}\PY{l+m+mi}{2}\PY{p}{)}
\PY{n}{r1} \PY{o}{=} \PY{n}{Rec}\PY{p}{(}\PY{p}{\PYZob{}}\PY{l+s+s1}{\PYZsq{}}\PY{l+s+s1}{l1}\PY{l+s+s1}{\PYZsq{}}\PY{p}{:}\PY{l+s+s1}{\PYZsq{}}\PY{l+s+s1}{a}\PY{l+s+s1}{\PYZsq{}}\PY{p}{,}\PY{l+s+s1}{\PYZsq{}}\PY{l+s+s1}{l2}\PY{l+s+s1}{\PYZsq{}}\PY{p}{:}\PY{l+s+s1}{\PYZsq{}}\PY{l+s+s1}{b}\PY{l+s+s1}{\PYZsq{}}\PY{p}{\PYZcb{}}\PY{p}{)}
\PY{n+nb}{print}\PY{p}{(}\PY{n}{show}\PY{p}{(}\PY{n}{Tr1}\PY{o}{.}\PY{n}{query}\PY{p}{(}\PY{n}{r1}\PY{p}{)}\PY{p}{)}\PY{p}{)}
\end{Verbatim}
\end{tcolorbox}

    \begin{Verbatim}[commandchars=\\\{\}]
0.06
\end{Verbatim}

    A simple dependent record type.

    \begin{tcolorbox}[breakable, size=fbox, boxrule=1pt, pad at break*=1mm,colback=cellbackground, colframe=cellborder]
\prompt{In}{incolor}{53}{\hspace{4pt}}
\begin{Verbatim}[commandchars=\\\{\}]
\PY{n}{dog} \PY{o}{=} \PY{n}{Pred}\PY{p}{(}\PY{l+s+s1}{\PYZsq{}}\PY{l+s+s1}{dog}\PY{l+s+s1}{\PYZsq{}}\PY{p}{,}\PY{p}{[}\PY{n}{Ind}\PY{p}{]}\PY{p}{)}
\PY{n}{a\PYZus{}dog} \PY{o}{=} \PY{n}{RecType}\PY{p}{(}\PY{p}{\PYZob{}}\PY{l+s+s1}{\PYZsq{}}\PY{l+s+s1}{x}\PY{l+s+s1}{\PYZsq{}}\PY{p}{:}\PY{n}{Ind}\PY{p}{,}\PY{l+s+s1}{\PYZsq{}}\PY{l+s+s1}{e}\PY{l+s+s1}{\PYZsq{}}\PY{p}{:}\PY{p}{(}\PY{n}{Fun}\PY{p}{(}\PY{l+s+s1}{\PYZsq{}}\PY{l+s+s1}{v}\PY{l+s+s1}{\PYZsq{}}\PY{p}{,} \PY{n}{Ind}\PY{p}{,} \PY{n}{PType}\PY{p}{(}\PY{n}{dog}\PY{p}{,}\PY{p}{[}\PY{l+s+s1}{\PYZsq{}}\PY{l+s+s1}{v}\PY{l+s+s1}{\PYZsq{}}\PY{p}{]}\PY{p}{)}\PY{p}{)}\PY{p}{,}\PY{p}{[}\PY{l+s+s1}{\PYZsq{}}\PY{l+s+s1}{x}\PY{l+s+s1}{\PYZsq{}}\PY{p}{]}\PY{p}{)}\PY{p}{\PYZcb{}}\PY{p}{)}
\PY{n}{show\PYZus{}latex}\PY{p}{(}\PY{n}{a\PYZus{}dog}\PY{p}{)}
\end{Verbatim}
\end{tcolorbox}
 
            
\prompt{Out}{outcolor}{53}{}
    
    \begin{equation}\left[\begin{array}{lcl}
\text{x} &:& \textit{Ind}\\
\text{e} &:& \langle \lambda v:\textit{Ind}\ .\ \text{dog}(v), \langle \text{x}\rangle\rangle
\end{array}\right]\end{equation}

    

    \begin{tcolorbox}[breakable, size=fbox, boxrule=1pt, pad at break*=1mm,colback=cellbackground, colframe=cellborder]
\prompt{In}{incolor}{54}{\hspace{4pt}}
\begin{Verbatim}[commandchars=\\\{\}]
\PY{n}{Ind}\PY{o}{.}\PY{n}{judge}\PY{p}{(}\PY{l+s+s1}{\PYZsq{}}\PY{l+s+s1}{d}\PY{l+s+s1}{\PYZsq{}}\PY{p}{)}
\PY{n}{PType}\PY{p}{(}\PY{n}{dog}\PY{p}{,}\PY{p}{[}\PY{l+s+s1}{\PYZsq{}}\PY{l+s+s1}{d}\PY{l+s+s1}{\PYZsq{}}\PY{p}{]}\PY{p}{)}\PY{o}{.}\PY{n}{judge}\PY{p}{(}\PY{l+s+s1}{\PYZsq{}}\PY{l+s+s1}{s1}\PY{l+s+s1}{\PYZsq{}}\PY{p}{,}\PY{o}{.}\PY{l+m+mi}{7}\PY{p}{)}
\PY{n}{r2} \PY{o}{=} \PY{n}{Rec}\PY{p}{(}\PY{p}{\PYZob{}}\PY{l+s+s1}{\PYZsq{}}\PY{l+s+s1}{x}\PY{l+s+s1}{\PYZsq{}}\PY{p}{:}\PY{l+s+s1}{\PYZsq{}}\PY{l+s+s1}{d}\PY{l+s+s1}{\PYZsq{}}\PY{p}{,}\PY{l+s+s1}{\PYZsq{}}\PY{l+s+s1}{e}\PY{l+s+s1}{\PYZsq{}}\PY{p}{:}\PY{l+s+s1}{\PYZsq{}}\PY{l+s+s1}{s1}\PY{l+s+s1}{\PYZsq{}}\PY{p}{,}\PY{l+s+s1}{\PYZsq{}}\PY{l+s+s1}{z}\PY{l+s+s1}{\PYZsq{}}\PY{p}{:}\PY{l+s+s1}{\PYZsq{}}\PY{l+s+s1}{other\PYZus{}stuff}\PY{l+s+s1}{\PYZsq{}}\PY{p}{\PYZcb{}}\PY{p}{)}
\PY{n}{show}\PY{p}{(}\PY{n}{a\PYZus{}dog}\PY{o}{.}\PY{n}{query}\PY{p}{(}\PY{n}{r2}\PY{p}{)}\PY{p}{)}
\end{Verbatim}
\end{tcolorbox}

            \begin{tcolorbox}[breakable, boxrule=.5pt, size=fbox, pad at break*=1mm, opacityfill=0]
\prompt{Out}{outcolor}{54}{\hspace{3.5pt}}
\begin{Verbatim}[commandchars=\\\{\}]
'0.7'
\end{Verbatim}
\end{tcolorbox}
        
    A simple record type with a path leading to another record type

    \begin{tcolorbox}[breakable, size=fbox, boxrule=1pt, pad at break*=1mm,colback=cellbackground, colframe=cellborder]
\prompt{In}{incolor}{55}{\hspace{4pt}}
\begin{Verbatim}[commandchars=\\\{\}]
\PY{n}{Tr2} \PY{o}{=} \PY{n}{RecType}\PY{p}{(}\PY{p}{\PYZob{}}\PY{l+s+s1}{\PYZsq{}}\PY{l+s+s1}{x}\PY{l+s+s1}{\PYZsq{}}\PY{p}{:}\PY{n}{a\PYZus{}dog}\PY{p}{\PYZcb{}}\PY{p}{)}
\PY{n}{show\PYZus{}latex}\PY{p}{(}\PY{n}{Tr2}\PY{p}{)}
\end{Verbatim}
\end{tcolorbox}
 
            
\prompt{Out}{outcolor}{55}{}
    
    \begin{equation}\left[\begin{array}{lcl}
\text{x} &:& \left[\begin{array}{lcl}
\text{x} &:& \textit{Ind}\\
\text{e} &:& \langle \lambda v:\textit{Ind}\ .\ \text{dog}(v), \langle \text{x}\rangle\rangle
\end{array}\right]
\end{array}\right]\end{equation}

    

    \begin{tcolorbox}[breakable, size=fbox, boxrule=1pt, pad at break*=1mm,colback=cellbackground, colframe=cellborder]
\prompt{In}{incolor}{56}{\hspace{4pt}}
\begin{Verbatim}[commandchars=\\\{\}]
\PY{n}{r3} \PY{o}{=} \PY{n}{Rec}\PY{p}{(}\PY{p}{\PYZob{}}\PY{l+s+s1}{\PYZsq{}}\PY{l+s+s1}{x}\PY{l+s+s1}{\PYZsq{}}\PY{p}{:}\PY{n}{r2}\PY{p}{\PYZcb{}}\PY{p}{)}
\PY{n}{show}\PY{p}{(}\PY{n}{Tr2}\PY{o}{.}\PY{n}{query}\PY{p}{(}\PY{n}{r3}\PY{p}{)}\PY{p}{)}
\end{Verbatim}
\end{tcolorbox}

            \begin{tcolorbox}[breakable, boxrule=.5pt, size=fbox, pad at break*=1mm, opacityfill=0]
\prompt{Out}{outcolor}{56}{\hspace{3.5pt}}
\begin{Verbatim}[commandchars=\\\{\}]
'0.7'
\end{Verbatim}
\end{tcolorbox}
        
    A slightly more complex record type.

    \begin{tcolorbox}[breakable, size=fbox, boxrule=1pt, pad at break*=1mm,colback=cellbackground, colframe=cellborder]
\prompt{In}{incolor}{57}{\hspace{4pt}}
\begin{Verbatim}[commandchars=\\\{\}]
\PY{n}{bark} \PY{o}{=} \PY{n}{Pred}\PY{p}{(}\PY{l+s+s1}{\PYZsq{}}\PY{l+s+s1}{bark}\PY{l+s+s1}{\PYZsq{}}\PY{p}{,}\PY{p}{[}\PY{n}{Ind}\PY{p}{]}\PY{p}{)}
\PY{n}{a\PYZus{}dog\PYZus{}bark} \PY{o}{=} \PY{n}{RecType}\PY{p}{(}\PY{p}{\PYZob{}}\PY{l+s+s1}{\PYZsq{}}\PY{l+s+s1}{x}\PY{l+s+s1}{\PYZsq{}}\PY{p}{:}\PY{n}{a\PYZus{}dog}\PY{p}{,}\PY{l+s+s1}{\PYZsq{}}\PY{l+s+s1}{e}\PY{l+s+s1}{\PYZsq{}}\PY{p}{:}\PY{p}{(}\PY{n}{Fun}\PY{p}{(}\PY{l+s+s1}{\PYZsq{}}\PY{l+s+s1}{v}\PY{l+s+s1}{\PYZsq{}}\PY{p}{,}\PY{n}{Ind}\PY{p}{,}\PY{n}{PType}\PY{p}{(}\PY{n}{bark}\PY{p}{,}\PY{p}{[}\PY{l+s+s1}{\PYZsq{}}\PY{l+s+s1}{v}\PY{l+s+s1}{\PYZsq{}}\PY{p}{]}\PY{p}{)}\PY{p}{)}\PY{p}{,}\PY{p}{[}\PY{l+s+s1}{\PYZsq{}}\PY{l+s+s1}{x.x}\PY{l+s+s1}{\PYZsq{}}\PY{p}{]}\PY{p}{)}\PY{p}{\PYZcb{}}\PY{p}{)}
\PY{n}{show\PYZus{}latex}\PY{p}{(}\PY{n}{a\PYZus{}dog\PYZus{}bark}\PY{p}{)}
\end{Verbatim}
\end{tcolorbox}
 
            
\prompt{Out}{outcolor}{57}{}
    
    \begin{equation}\left[\begin{array}{lcl}
\text{x} &:& \left[\begin{array}{lcl}
\text{x} &:& \textit{Ind}\\
\text{e} &:& \langle \lambda v:\textit{Ind}\ .\ \text{dog}(v), \langle \text{x}\rangle\rangle
\end{array}\right]\\
\text{e} &:& \langle \lambda v:\textit{Ind}\ .\ \text{bark}(v), \langle \text{x.x}\rangle\rangle
\end{array}\right]\end{equation}

    

    \begin{tcolorbox}[breakable, size=fbox, boxrule=1pt, pad at break*=1mm,colback=cellbackground, colframe=cellborder]
\prompt{In}{incolor}{58}{\hspace{4pt}}
\begin{Verbatim}[commandchars=\\\{\}]
\PY{n}{PType}\PY{p}{(}\PY{n}{bark}\PY{p}{,}\PY{p}{[}\PY{l+s+s1}{\PYZsq{}}\PY{l+s+s1}{d}\PY{l+s+s1}{\PYZsq{}}\PY{p}{]}\PY{p}{)}\PY{o}{.}\PY{n}{judge}\PY{p}{(}\PY{l+s+s1}{\PYZsq{}}\PY{l+s+s1}{s2}\PY{l+s+s1}{\PYZsq{}}\PY{p}{,}\PY{o}{.}\PY{l+m+mi}{3}\PY{p}{)}
\PY{n}{r4} \PY{o}{=} \PY{n}{Rec}\PY{p}{(}\PY{p}{\PYZob{}}\PY{l+s+s1}{\PYZsq{}}\PY{l+s+s1}{x}\PY{l+s+s1}{\PYZsq{}}\PY{p}{:}\PY{n}{r2}\PY{p}{,}\PY{l+s+s1}{\PYZsq{}}\PY{l+s+s1}{e}\PY{l+s+s1}{\PYZsq{}}\PY{p}{:}\PY{l+s+s1}{\PYZsq{}}\PY{l+s+s1}{s2}\PY{l+s+s1}{\PYZsq{}}\PY{p}{\PYZcb{}}\PY{p}{)}
\PY{n}{show\PYZus{}latex}\PY{p}{(}\PY{n}{r4}\PY{p}{)}
\end{Verbatim}
\end{tcolorbox}
 
            
\prompt{Out}{outcolor}{58}{}
    
    \begin{equation}\left[\begin{array}{rcl}
\text{x} &=& \left[\begin{array}{rcl}
\text{x} &=& \text{d}\\
\text{e} &=& \text{s}_{\text{1}}\\
\text{z} &=& \text{other}_{\text{stuff}}
\end{array}\right]\\
\text{e} &=& \text{s}_{\text{2}}
\end{array}\right]\end{equation}

    

    \begin{tcolorbox}[breakable, size=fbox, boxrule=1pt, pad at break*=1mm,colback=cellbackground, colframe=cellborder]
\prompt{In}{incolor}{59}{\hspace{4pt}}
\begin{Verbatim}[commandchars=\\\{\}]
\PY{n}{show}\PY{p}{(}\PY{n}{a\PYZus{}dog\PYZus{}bark}\PY{o}{.}\PY{n}{query}\PY{p}{(}\PY{n}{r4}\PY{p}{)}\PY{p}{)}
\end{Verbatim}
\end{tcolorbox}

            \begin{tcolorbox}[breakable, boxrule=.5pt, size=fbox, pad at break*=1mm, opacityfill=0]
\prompt{Out}{outcolor}{59}{\hspace{3.5pt}}
\begin{Verbatim}[commandchars=\\\{\}]
'0.21'
\end{Verbatim}
\end{tcolorbox}
        
    Conditional probabilities.

    \begin{tcolorbox}[breakable, size=fbox, boxrule=1pt, pad at break*=1mm,colback=cellbackground, colframe=cellborder]
\prompt{In}{incolor}{60}{\hspace{4pt}}
\begin{Verbatim}[commandchars=\\\{\}]
\PY{n}{show}\PY{p}{(}\PY{n}{a\PYZus{}dog\PYZus{}bark}\PY{o}{.}\PY{n}{query}\PY{p}{(}\PY{n}{r4}\PY{p}{,}\PY{p}{[}\PY{p}{(}\PY{l+s+s1}{\PYZsq{}}\PY{l+s+s1}{s2}\PY{l+s+s1}{\PYZsq{}}\PY{p}{,}\PY{n}{PType}\PY{p}{(}\PY{n}{bark}\PY{p}{,}\PY{p}{[}\PY{l+s+s1}{\PYZsq{}}\PY{l+s+s1}{d}\PY{l+s+s1}{\PYZsq{}}\PY{p}{]}\PY{p}{)}\PY{p}{)}\PY{p}{]}\PY{p}{)}\PY{p}{)}
\end{Verbatim}
\end{tcolorbox}

            \begin{tcolorbox}[breakable, boxrule=.5pt, size=fbox, pad at break*=1mm, opacityfill=0]
\prompt{Out}{outcolor}{60}{\hspace{3.5pt}}
\begin{Verbatim}[commandchars=\\\{\}]
'0.7'
\end{Verbatim}
\end{tcolorbox}
        
    \begin{tcolorbox}[breakable, size=fbox, boxrule=1pt, pad at break*=1mm,colback=cellbackground, colframe=cellborder]
\prompt{In}{incolor}{61}{\hspace{4pt}}
\begin{Verbatim}[commandchars=\\\{\}]
\PY{n}{show}\PY{p}{(}\PY{n}{a\PYZus{}dog\PYZus{}bark}\PY{o}{.}\PY{n}{query}\PY{p}{(}\PY{n}{r4}\PY{p}{,}\PY{p}{[}\PY{p}{(}\PY{l+s+s1}{\PYZsq{}}\PY{l+s+s1}{s1}\PY{l+s+s1}{\PYZsq{}}\PY{p}{,}\PY{n}{PType}\PY{p}{(}\PY{n}{dog}\PY{p}{,}\PY{p}{[}\PY{l+s+s1}{\PYZsq{}}\PY{l+s+s1}{d}\PY{l+s+s1}{\PYZsq{}}\PY{p}{]}\PY{p}{)}\PY{p}{)}\PY{p}{,}\PY{p}{(}\PY{l+s+s1}{\PYZsq{}}\PY{l+s+s1}{s2}\PY{l+s+s1}{\PYZsq{}}\PY{p}{,}\PY{n}{PType}\PY{p}{(}\PY{n}{bark}\PY{p}{,}\PY{p}{[}\PY{l+s+s1}{\PYZsq{}}\PY{l+s+s1}{d}\PY{l+s+s1}{\PYZsq{}}\PY{p}{]}\PY{p}{)}\PY{p}{)}\PY{p}{]}\PY{p}{)}\PY{p}{)}
\end{Verbatim}
\end{tcolorbox}

            \begin{tcolorbox}[breakable, boxrule=.5pt, size=fbox, pad at break*=1mm, opacityfill=0]
\prompt{Out}{outcolor}{61}{\hspace{3.5pt}}
\begin{Verbatim}[commandchars=\\\{\}]
'1.0'
\end{Verbatim}
\end{tcolorbox}
        

    % Add a bibliography block to the postdoc
    
    
    
    \end{document}

%%% Local Variables:
%%% mode: latex
%%% TeX-master: t
%%% End:
